\documentclass[12pt,english]{reviewresponse}

%% Language
\usepackage{babel}
\usepackage[babel]{microtype}
\usepackage[babel]{csquotes}


%% Fonts
\usepackage[T1]{fontenc}
\usepackage{lmodern}
%\usepackage{newcent} % different font
%\usepackage[scaled]{beramono} % different monospace font

%% Crossreferences
\usepackage{xr-hyper}
\usepackage{hyperref}
\usepackage{refcount}

% Define a dummy command so the file doesn't crash when reading AGU citations
\providecommand{\APACbibcite}[2]{}


\externaldocument{agujournaltemplate}

\usepackage{xparse} % Provides the advanced argument handling

% Syntax: \lineref{start_label}[end_label]
% m = mandatory argument (start)
% o = optional argument (end, in brackets)
\NewDocumentCommand{\lineref}{m o}{%
  \IfNoValueTF{#2}{%
    % CASE 1: Only one argument provided -> \lineref{label}
    line \ref{#1}%
  }{%
    % CASE 2: Two arguments provided -> \lineref{start}[end]
    \ifnum\getrefnumber{#1}=\getrefnumber{#2}%
       line \ref{#1}% Same line
    \else%
       lines \ref{#1}--\ref{#2}% Different lines
    \fi%
  }%
}


%% Bibliography
\usepackage[backend=biber,style=ieee,dashed=false,url=false,isbn=false,defernumbers=true,refsection=section]{biblatex}
\bibliography{literature.bib}

\usepackage{hyperref}


\title{Reversible Regime Change: Climate-driven phytoplankton community shifts in the Cariaco Basin, Venezuela}
\author{Benjamin Post, Esteban Acevedo-Trejos, Subhendu Chakraborty, Andrew D. Barton and Agostino Merico}
\journal{Journal of Geophysical Research - Biogeosciences}
\manuscript{\#2025JG009360}
\editorname{Marguerite A. Xenopoulos}

\begin{document}
\maketitle

% Cover Letter
\input{cover_letter.tex}

% Response to Editor
\editor
\begin{generalcomment}
	This manuscript has been examined by two expert reviewers. Overall, both are supportive of this work, but highlight some important concerns related to lower resolution of observations toward the end of the time-series, leading to possible seasonal biases. The impact of this possible bias on the conclusions is unclear. I urge the authors to pursue the recommendations from the reviewers in order to fortify their analyses.
\end{generalcomment}
\begin{revresponse}[We appreciate your handling of the review process.]
	According to the reviewers' comments, we have checked our manuscript and addressed them in the following way:
	\begin{enumerate}
		\item We added an in-depth discussion of the impact of missing data, particularly towards the end of the time series.
		\item We corrected spelling errors etc...
	\end{enumerate}
\end{revresponse}
\begin{concludingresponse}[to the Editor]
	Thank you for your valuable comments on our manuscript. 
	We have done our best to incorporate changes to reflect the suggestions, which allowed us to improve the quality of our work.
\end{concludingresponse}


% Reviewer 1
\reviewer
\begin{generalcomment}
	This manuscript examines climatic, physical and biogeochemical drivers controlling multi-year variability of phytoplankton community composition in the Cariaco Basin, Venezuela, using a diverse set of environmental predictors including climate indices like the Atlantic Multidecadal Oscillation (AMO) and the El Niño Southern Oscillation (MEI v.2), atmospheric layers like wind speed, precipitation and evaporation from a reanalysis model, and concurrent in situ properties collected from CTD casts including seawater temperature, salinity, and dissolved inorganic nutrients. These observations are paired with monthly chlorophyll-a and micro-phytoplankton cell counts performed with microscopy collected from Niskin samples at multiple depths within the upper 100 m of the water column by the CARIACO Ocean Time Series project between 1996 and 2016. The authors compute yearly anomalies in predictor observations and phytoplankton densities using normalized "z" scores to identify trends in physical and biogeochemical parameters and detect shifts in community composition as well as in diversity metrics (i.e. species richness, Shannon index, Pielou's eveness) of primary producers. These analyses are further assisted by multi-dimensional ordination techniques. The authors apply a random forest model with gradient analysis to evaluate correlations between predictor metrics and phytoplankton assemblages and quantify the relative importance of each environmental parameter in shaping species composition of primary producers in the Cariaco Basin. The authors find a strong inverse correlation between AMO and wind speed, and phytoplankton counts, especially diatoms and haptophytes, attributed to changes in Ekman upwelling and associated nutrient supply to the euphotic zone. Clear seasonal and interannual trends in species richness emerged from the 20-year times series but not for all other diversity indices.
\end{generalcomment}

\begin{generalcomment} (continued) 
Results show a drastic shift in the phytoplankton community in 2004 coinciding with an AMO sign shift that lasted through 2013, followed by an assemblage recovery in 2014-2016 when the AMO sign returned to negative values. Gradient forest analysis indicated that AMO has the strongest effect on Cariaco's phytoplankton community structure followed by SST, MEI v.2, and nitrate concentration. The goals of the study are clear, the methodological approach is sound, and data analyses are novel and interesting, and results contribute valuable insights into bottom-up controls on primary producers in the Cariaco Basin that can in turn help to better resolve the rich paleo-climate record preserved in the basin's sediments. The manuscript is well written and concise.
\end{generalcomment}

\begin{revresponse}[Thank you for your feedback.]
	We have carefully ....
\end{revresponse}

\begin{revcomment}
	The Cariaco Basin is subject to a secondary upwelling pulse that typically occurs between June and August every year. This event is mainly controlled by the intensification of wind curl and transport of the Caribbean current, and not by ITCZ dynamics. While this upwelling event is somewhat weaker than the primary upwelling during January-April it can lead to significant shoaling of the nutricline and associated increase in primary productivity, chl-a concentration, and diatom abundance. This secondary upwelling pulse likely adds nuance to the seasonality of the phytoplankton community that cannot be captured with annually averaged data due to signal smoothing and attenuation. Therefore, there could value in performing analyses with seasonally aggregated data to determine whether the phytoplankton response to the AMO shift is dominated solely by changes in primary upwelling or if this effect also becomes apparent as changes in the strength of the secondary upwelling. Seasonal data analysis could aid in 1) quantifying the impact of shifts in AMO and MEI v.2 on this secondary welling pulse, 2) identifying additional community clusters and their association to a particular index, and 3) disentangling physical drivers of species turnover (as shown in Figure 4) during primary and secondary upwelling events.
\end{revcomment}
\begin{revresponse}
	We thank the reviewer for this insightful suggestion. We agree that the secondary upwelling (June--August) represents a distinct oceanographic event with potentially different drivers than the primary upwelling (December--May). To address this, we conducted additional analyses stratifying the data by three seasons: primary upwelling, secondary upwelling, and rainy season (September--November).
    
    \textit{Seasonal gradient forest analysis}
    
    We ran gradient forest models separately for each season (primary upwelling $n=104$, secondary upwelling $n=47$, rainy season $n=42$). Given modest sample sizes for the latter two seasons, we interpret these results as exploratory.
    
    Key findings:
    \begin{itemize}
        \item AMO ranked as the top predictor across all three seasons with optimized time lags. Notably, AMO emerged as the strongest predictor for secondary upwelling even without any time lag, indicating a strong linkage to community composition during this period. The optimal lag was shorter for secondary upwelling (2 months) than for primary upwelling (3 months), potentially suggesting a more immediate link between AMO and community response during secondary upwelling, although this one-month difference is marginal in the context of a multi-decadal time series.
        \item MEI v.2 ranked second for both primary and secondary upwelling, suggesting ENSO teleconnections influence both upwelling events.
        \item Silicate and total precipitation emerged among the top predictors during secondary upwelling. Given that precipitation peaks during June--August (visible in seasonal density distributions), elevated silicate likely reflects enhanced surface runoff during this period rather than upwelling-driven nutrient supply.
        \item The composition of predictable genera varied by season: primary upwelling models predicted predominantly diatoms alongside dinoflagellates and cyanobacteria; secondary upwelling and rainy season models predicted mostly diatoms and dinoflagellates, with dinoflagellates particularly prominent during the rainy season.
    \end{itemize}
    
    These results suggest that while AMO remains the dominant driver year-round, secondary upwelling exhibits distinct predictor signatures consistent with runoff influence rather than wind-driven nutrient supply.
    
    \textit{Community clustering by season}
    
    To assess whether seasonal variation produces distinct community states, we examined NMDS ordination colored by season and performed permutational ANOVA (Adonis) with temporal cluster and season as factors.
    
    Results:
    \begin{itemize}
        \item Temporal cluster (interannual, AMO-related): $R^2 = 15.9\%$, $p = 0.001$
        \item Season: $R^2 = 3.7\%$, $p = 0.001$
    \end{itemize}
    
    While seasonality explains statistically significant variance, its effect size is approximately four times smaller than the interannual signal from our identified clusters. The NMDS ordination does not reveal a distinct community state associated with secondary upwelling (Figure~SX, Table~SX). This suggests that annually aggregated analysis captures the dominant mode of community variability.
    
    \textit{Limitations}
    
    We lack direct observations of ITCZ position and Caribbean current transport. Our wind speed data (ERA5 u10) does not capture wind curl. Future work incorporating these variables could further elucidate the distinct forcing of secondary upwelling.
    
    \textit{Conclusion}
    
    Our seasonal analyses confirm that AMO is the strongest predictor during both upwelling events, secondary upwelling shows distinct predictor signatures (silicate, precipitation) consistent with runoff influence, and the seasonal signal in community composition is substantially weaker than the interannual signal. These findings reinforce our main conclusion that decadal-scale AMO variability is the dominant driver of phytoplankton community shifts, while adding nuance regarding seasonal expression of this influence.

    % NOTE: I need to add the figures to the supplementary, and reference them here, then also add a discussion of this to the actual manuscript!
    
\end{revresponse}

\begin{revcomment}
Trichodesmium and other cyanobacteria likely have a seasonal signal in the Basin with higher values in the non-upwelling period, which is characterized by strong stratification of the water column, lower nutrient and chl-a concentrations, and conditions that are generally favorable to nitrogen fixers. Higher abundances of Trichodesmium and Synechococcus during the "cluster 2" period, or years with positive AMO, weaker nutrient supply and stronger stratification, can be expected and especially if data are aggregated by season. It may be worth exploring this hypothesis.
\end{revcomment}
\begin{revresponse}
	We thank the reviewer for this insightful suggestion regarding cyanobacteria dynamics. We tested the hypothesis that cyanobacteria abundances, particularly \textit{Trichodesmium} and \textit{Synechococcus}, would be higher during stratified conditions (Cluster~2, positive AMO) by examining genus-level patterns.

    At the aggregate level, cyanobacteria cell counts from microscopy were significantly higher in Cluster~1 than in Cluster~2 across all seasons (Wilcoxon tests: Upwelling $p < 0.001$, Secondary Upwelling $p < 0.001$, Rainy $p = 0.043$). However, genus-level analysis revealed contrasting patterns between the two dominant cyanobacteria taxa.

    \textit{Trichodesmium} showed no significant difference between clusters (Wilcoxon $p = 0.11$) but exhibited strong seasonal variation within both clusters (Kruskal-Wallis: Cluster~1 $\chi^2 = 14.7$, $p < 0.001$; Cluster~2 $\chi^2 = 22.8$, $p < 0.001$), with highest abundances during the secondary upwelling and rainy seasons. Correlations with stratification indicators supported the reviewer's hypothesis: \textit{Trichodesmium} was positively correlated with SST ($\rho = 0.48$, $p < 0.001$), 21\,°C isotherm depth ($\rho = 0.43$, $p < 0.001$), and AMO ($\rho = 0.24$, $p < 0.001$), and negatively correlated with nitrate concentration ($\rho = -0.39$, $p < 0.001$). These results are consistent with the ecology of \textit{Trichodesmium} as a nitrogen-fixer favored by warm, stratified, nutrient-depleted conditions.

    In contrast, \textit{Synechococcus} was detected exclusively during Cluster~1, with zero counts recorded throughout the entire Cluster~2 period (Wilcoxon $p < 0.001$). This pattern is opposite to ecological expectations, as \textit{Synechococcus} typically thrives under stratified, oligotrophic conditions. The absence of \textit{Synechococcus} detections in Cluster~2 likely reflects the methodological constraints of light microscopy rather than true absence. \textit{Synechococcus} cells are approximately 1\,\textmu m in diameter, well below the $\sim$20\,\textmu m detection limit of the Utermöhl method employed in this study. The detections recorded during Cluster~1 may represent larger colonial aggregates or cells associated with detrital particles that were visible during high-biomass conditions.

    These findings highlight the complementary value of flow cytometry or pigment-based approaches (e.g., zeaxanthin) for characterizing picoplanktonic cyanobacteria responses to stratification, which was beyond the scope of the present analysis focused on the microscopy cell count data. Nevertheless, the \textit{Trichodesmium} results provide partial support for the reviewer's hypothesis regarding nitrogen-fixer dynamics in the Cariaco Basin.
\end{revresponse}


\begin{revcomment}
Monthly primary productivity rate (PP) measurements derived from 14C incubations were also performed throughout the duration of the CARIACO project. It may be worth examining climate indices and physical parameters as predictors of PP, and in turn relationships between PP and phytoplankton community clusters, possibly yielding interesting insights into the roles of different assemblages in PP variability and carbon sequestration.
\end{revcomment}
\begin{revresponse}
	We thank the reviewer for this thoughtful suggestion regarding primary productivity (PP) measurements. We investigated the relationship between PP and chlorophyll~\textit{a} concentration, finding a strong positive correlation (Spearman's $\rho = 0.70$, $p < 0.001$, $n = 202$; Pearson's $r = 0.74$ on log-transformed values). This indicates that chlorophyll~\textit{a} captures a substantial portion of PP variability ($R^2 = 0.55$).

    Examining PP correlations with the key predictors identified in our gradient forest analysis, we found that PP was significantly correlated with SST ($\rho = -0.60$), nitrate concentration ($\rho = 0.60$), and 21\,°C isotherm depth ($\rho = -0.55$), consistent with bottom-up control through the upwelling system. The correlation with AMO was weaker but significant ($\rho = -0.18$, $p < 0.01$), while MEI~v.2 showed no significant relationship with PP ($\rho = 0.04$, $p = 0.54$).

    Interestingly, when comparing PP between the two community clusters, we found no significant difference (Wilcoxon test, $W = 5487$, $p = 0.27$), despite chlorophyll~\textit{a} differing significantly between clusters ($p < 0.001$). This suggests that community-level productivity was maintained across the regime shift, possibly due to the compensatory contributions of smaller phytoplankton cells that were below the detection limit of light microscopy but captured by the \textsuperscript{14}C incubations. This observation is consistent with our interpretation of a shift toward smaller cell sizes during the low-upwelling period.

    However, the PP dataset contains substantial gaps, with 20\% of monthly observations missing overall and coverage declining markedly after 2009. Given the strong covariance between PP and chlorophyll~\textit{a}, the similar relationships with upwelling-related predictors, and the data coverage limitations, we believe chlorophyll~\textit{a} adequately represents productivity dynamics for the purposes of our community analysis.
\end{revresponse}


\begin{revcomment}
Years in which shifts in environmental conditions and phytoplankton are observed mentioned throughput the text often don't match what is shown on figures. This is confusing and makes the reading and interpretation of results difficult. A few instances of where in the manuscript such mismatches occur are specified in the "Editorial comments" section below but please review this carefully.
\end{revcomment}
\begin{revresponse}
	We have carefully revised the manuscript, and corrected the mismatches that are mentioned in the comments.
\end{revresponse}



\begin{revcomment}
Line 145: "This area is relatively productive ($\sim 400$ gCm$^{-2}$ y$^{-1}$)". How was this productivity estimated? If this is an average, add standard deviation value.
\end{revcomment}
\begin{revresponse}
	This value was chosen as a representative value based on reported values from literature, thank you for highlighting the missing reference and explanation.
    We have now calculated the average primary productivity across the entire time series using the raw data and added the exact mean value and standard deviation (\lineref{R1:C5:start}[R1:C5:end]):
    \begin{changes}
       This area is relatively productive (mean integrated primary production across the entire time series of $485 \pm 345$ g C m$^{-2}$ y$^{-1}$),...
    \end{changes}
    The calculation is based on the reported depth-discrete measurements of $^{14}$C-incubation based NPP with the units mg C m$^{-3}$ h$^{-1}$. These are depth integrated to 100 m, multiplied by 12 hours of daylight and 365 days in a year, and divided by 1000 to yield g C m$^{-2}$ y$^{-1}$. Mean and standard deviation were calculated with the depth-integrated values of all available monthly cruises.
\end{revresponse}

\begin{revcomment}
Lines 147-148: The main reason for anoxic conditions is lack of ventilation of waters below the sill, not just high PP.
\end{revcomment}
\begin{revresponse}
	We have added this point to the sentence in \lineref{R1:C6}.
\end{revresponse}

\begin{revcomment}
Line 168: Consider rewording this sentence to indicate an overlap of 8 nutrient samples between both labs, if this is what was intended to say.
\end{revcomment}
\begin{revresponse}
	We have adjusted the wording according to your suggestion (see \lineref{R1:C7}).
\end{revresponse}

\begin{revcomment}
Lines 169-170: This sentence is not clear. Was this done for the overlapping nutrient measurements only?
\end{revcomment}
\begin{revresponse}
	We added the clarification to this line, that only the overlapping nutrient measurements were averaged.
\end{revresponse}

\begin{revcomment}
Line 200: "Cariaco time series" should be replaced by "CARIACO Time Series" for consistency.
\end{revcomment}
\begin{revresponse}
	This was corrected.
\end{revresponse}

\begin{revcomment}
Lines 203-205: Why was evaporation selected as a variable? How does evaporation control phytoplankton communities?
\end{revcomment}
\begin{revresponse}
	Evaporation was initially included as a proxy for heat flux, as it is driven by wind speed, sea surface temperature, and atmospheric humidity conditions. However, we agree with the reviewer that its inclusion is not well justified from a mechanistic perspective. Unlike precipitation, which has a direct influence on phytoplankton communities through its association with surface runoff, evaporation lacks a clear pathway of influence on community composition. Furthermore, its strong covariance with sea surface temperature and salinity means that any climate-related variance it might capture is already represented by these variables. We have therefore removed evaporation from the analysis. This modification did not qualitatively affect our results.
\end{revresponse}

\begin{revcomment}
Line 213: Delete "sea surface temperature" and use SST only.
\end{revcomment}
\begin{revresponse}
	This was corrected.
\end{revresponse}

\begin{revcomment}
Line 238: Note that the Results section make reference to data from 2016.
\end{revcomment}
\begin{revresponse}
	This section of the Methods is only in reference to Figure 2 and the analysis of interannual variance using z-scores. For this specific analysis the years 1995, 2016, 2017 were excluded, for other parts of our analysis (where no annual averaging was performed) the monthly measurements from those years were included.
    
    We have added the following paragraph in \lineref{R1:C12:start}[R1:C12:end] to clarify this:
    \begin{changes}
        For the analysis of annual z-scores, we excluded years with fewer than six monthly observations. This applied to 1995 (n=2; November–December only), 2016 (n=5; January, February, May, September, December), and 2017 (n=1; January only). Climate and meteorological variables (AMO, MEI v.2, wind speed, precipitation, evaporation) had complete monthly coverage from 1996–2016, while in-situ measurements showed occasional gaps throughout the time series, with sporadic missing months in 1997–2013 (typically 1–3 months per year) and more substantial gaps in 2014 (3 months missing: March, July, August) and 2015 (4 months missing: January, May, June, October). Importantly, despite these gaps, both 2014 and 2015 retained observations from both upwelling (December–May) and rainy (June–November) seasons, reducing the risk of seasonal bias in annual averages.
    \end{changes}
\end{revresponse}

\begin{revcomment}
Line 258: A period after "matrix" is missing.
\end{revcomment}
\begin{revresponse}
	This was corrected.
\end{revresponse}

\begin{revcomment}
Lines 327-329: Note that z scores for precipitation and salinity have an inverse relationship with silicates, indicating increased runoff.
\end{revcomment}
\begin{revresponse}
	We have added this information to the paragraph (see \lineref{R1:C14}):
    \begin{changes}
        Silicate anomalies, in contrast, did not follow the same pattern as nitrate and phosphate. Instead, annual anomalies show an inverse relationship with precipitation and salinity, indicating effects of surface runoff. Elevated concentrations were observed between 2008 and 2013, ...
    \end{changes}
\end{revresponse}

\begin{revcomment}
Lines 332-334: This should be reworded to specify that the persisting shift occurs for AMO and some physical parameters, and not in chl-a or cell counts for which the shift happens later, in 2005.
\end{revcomment}
\begin{revresponse}
	We reworded the paragraph (\lineref{R1:C15:start}[R1:C15:end]) to clarify this:
    \begin{changes}
        This biological shift was preceded by a change in the AMO index anomalies that occurred after 2002, when the index transitioned to a period of mostly positive anomalies that persisted until 2013. This period was also characterized by upwelling-related variables indicating a weak upwelling regime (low wind, high SST, deep isotherm). Thus, while the AMO and physical parameters shifted after 2002, the corresponding change in chlorophyll a and cell counts occurred later, after 2004.
    \end{changes}
\end{revresponse}

\begin{revcomment}
Lines 338-340: Consider rewording to "Shannon diversity z scores increase between 2007 and 2013, when chl-a and cell counts were generally low".
\end{revcomment}
\begin{revresponse}
	We incorporated your suggestion (see \lineref{R1:C16}).
\end{revresponse}

\begin{revcomment}
Lines 348-351: The figure shows chl-a within 0-25m lower than in the 25-50m layer occurring in 1999, 2005, 2008, 2009, and 2013. Years in the text don't match what is shown in the figure.
\end{revcomment}
\begin{revresponse}
	We have corrected the years mentioned.
\end{revresponse}

\begin{revcomment}
Line 356: Delete "In fall". There is no fall season in the tropics as in higher latitudes.
\end{revcomment}
\begin{revresponse}
	We deleted "in fall".
\end{revresponse}

\begin{revcomment}
Line 362: Delete "in winter".
\end{revcomment}
\begin{revresponse}
	We deleted "in winter".
\end{revresponse}

\begin{revcomment}
Line 371: These events happened in 2006, 2009 and 2015 (not 2016). Please double-check that the correct years are referenced throughout the manuscript.
\end{revcomment}
\begin{revresponse}
	We corrected the mentioned year from 2015 to 2016 and went through the entire manuscript to check for such errors.
\end{revresponse}

\begin{revcomment}
Lines 371-372: There is a seasonal signal for bulk chl-a and richness only. All other diversity metrics don't show a seasonal change.
\end{revcomment}
\begin{revresponse}
	We have adjusted the wording so that differentiation is clearer.
\end{revresponse}

\begin{revcomment}
Lines 376-378: Double-check these years.
\end{revcomment}
\begin{revresponse}
	We checked the years, and corrected where they were off.
\end{revresponse}

\begin{revcomment}
Line 439: Delete "dashed". Phytoplankton curves are all solid lines.
\end{revcomment}
\begin{revresponse}
	We corrected the linestyle descriptions. The overall cumulative importance is a dashed black line, the functional group specific cumulative importance is a solid colored line.
\end{revresponse}

\begin{revcomment}
Lines 440-441: Where in the figure is this shown? This statement is confusing because the cumulative importance increases toward positive AMO values.
\end{revcomment}
\begin{revresponse}
	We have added an explanatory paragraph (\lineref{R1:C24:start}[R1:C24:end]) before this assertion, to explain how to interpret the cumulative importance curves:
    \begin{changes}
        Cumulative importance curves from gradient forest analysis show how the importance of a predictor accumulates across its range of values. The slope of the curve at any point indicates where the strongest compositional turnover occurs—steep sections represent threshold values where small changes in the predictor correspond to large changes in community composition, while flat sections indicate regions where the predictor has little influence on community structure. For the AMO index, we see a clear pattern in which negative anomalies affect the community (slope is highest for negative values).
    \end{changes}
\end{revresponse}

\begin{revcomment}
Line 459: Use SST only, no need for "sea surface temperature".
\end{revcomment}
\begin{revresponse}
	We removed "sea surface temperature".
\end{revresponse}

\begin{revcomment}
Line 461: Use AMO only.
\end{revcomment}
\begin{revresponse}
	We corrected this.
\end{revresponse}

\begin{revcomment}
Lines 462-463: Weren't data from 2017 excluded from the study because of poor coverage?
\end{revcomment}
\begin{revresponse}
	We only removed data from the time series for annual averages. For the gradient forest analysis we use the data at it's original monthly sampling resolution and only exclude months with missing data.

\end{revresponse}

\begin{revcomment}
Line 498: There is a typo, change to "grouping".
\end{revcomment}
\begin{revresponse}
	We corrected the typo.
\end{revresponse}

\begin{revcomment}
Line 504: Double-check these years. A decline in richness seem to happen between 1996 and 2009.
\end{revcomment}
\begin{revresponse}
	We considered the decline from 1998 onwards, since annual z scores for 1996 and 1997 are actually positive. We corrected the end of the period to 2009.
\end{revresponse}

\begin{revcomment}
Lines 515-516: Upwelling should favor growth of fewer r-selected species dominating the community and therefore lower species evenness.
\end{revcomment}
\begin{revresponse}
	We have added this information to the paragraph.
\end{revresponse}

\begin{revcomment}
Lines 569-572: Enhanced eddy activity near the Cariaco Basin during strong El Niño can stimulate productivity and lead to changes in community composition. Ventilation of sub-oxic waters below the sill from eddy activity has been documented by Astor et al 2003. The basin is also subject to a mid-year secondary upwelling typically occurring in June-August induced by intensification of the Caribbean Current (see Rueda-Roa et al 2018). Stronger El Niño could be related to enhanced secondary upwelling but not necessarily to intensification of the primary upwelling during January-April. See: https://doi-org.access.library.miami.edu/10.1016/S0278-4343(02)00130-9 and https://doi.org/10.3390/jmse6020036.
\end{revcomment}
\begin{revresponse}
	Thank you for the very relevant addition. We have incorporated the information and citations to that paragraph.
\end{revresponse}

\begin{revcomment}
Lines 587-593: El Niño events typically result in low precipitation and extremely dry conditions in the southern Caribbean. The expectation is that higher silicate concentration and lower surface salinity are observed during strong negative MEI years (La Niña). I suggest exploring this relationship.
\end{revcomment}
\begin{revresponse}
	We have added a sentence discussing and mentioning this relationship to the paragraph.
\end{revresponse}

\begin{revcomment}
Line 625: There is a typo "?,?:".
\end{revcomment}
\begin{revresponse}
	We have corrected the type in the citation, now it correctly renders it as "Pitcher, Lawton, et al. 2012".
\end{revresponse}

\begin{revcomment}
Lines 632-638: Abundance of Spanish sardine in Cariaco can also be an important top-down control on phytoplankton. See Rueda-Roa et al 2017: https://doi.org/10.1371/journal.pone.0179984.
\end{revcomment}
\begin{revresponse}
	We have included this important information in the paragraph and cited Rueda-Roa et al. 2017.
\end{revresponse}

\begin{revcomment}
Figure 6: In the figure caption, change to "Dashed black line" and "Colored solid lines".
\end{revcomment}
\begin{revresponse}
	We corrected the figure caption and added the appropriate linestyle descriptions.
\end{revresponse}

\begin{revcomment}
Figure B2: Please clarify how this figure is different from Figure 6. Both figures have nitrate as predictor but the curves are different. It is not clear from the caption why this is the case. The same with AMO. Also, note that Trichodesmium belongs to cyanobacteria and is misclassified here (not in Figure 6).
\end{revcomment}
\begin{revresponse}
    We thank the reviewer for identifying this oversight. Figure C2 and C3 (in the update manuscript, previously these were B2 and B3) show the gradient forest analysis performed without any time lags in the predictor variables, whereas Figure 6 in the main text includes optimized time lags for climate and meteorological variables (see Table C1). This comparison was included to demonstrate the effect of incorporating time lags on predictor rankings and response curves, as the gradient forest algorithm showed sensitivity to this methodological choice. Notably, even without time lags, AMO and MEI v.2 rank relatively high among predictors, supporting the robustness of our finding that climate indices are important predictors of community change.
    
    The differences in the cumulative importance curves for nitrate (and other variables without time lags) between Figures 6 and C2 arise from two sources: (1) the stochastic nature of the random forest models underlying the gradient forest algorithm, and (2) the conditional permutation procedure, which accounts for correlations among predictors and can lead to different importance attributions when the predictor set changes (i.e., when lagged vs. non-lagged variables are used). The AMO curves differ more substantially because Figure 6 uses a 2-month lag while Figure C2 uses no lag.
    
    We have corrected the figure caption to explicitly state that these supplementary figures show results without time lags. We have also corrected the classification of \textit{Trichodesmium}, which was erroneously colored as "other" rather than a cyanobacterium in Figure C2. We thank the reviewer for catching this error.
\end{revresponse}

\begin{concludingresponse}[]
	Thank you for your valuable comments on our manuscript. 
\end{concludingresponse}

%%%%%%%%%%%%%%%%%%%%%%%%%%%%%%%%%%%%%%%%%%%%%%%%%
%%%%%%%%%%%%%%%%%%%%%%%%%%%%%%%%%%%%%%%%%%%%%%%%%
%%%%%%%%%%%%%%%%%%%%%%%%%%%%%%%%%%%%%%%%%%%%%%%%%
%%%%%%%%%%%%%%%%%%%%%%%%%%%%%%%%%%%%%%%%%%%%%%%%%

% Reviewer 2
\reviewer
\label{rev:2}
\begin{generalcomment}
	The paper describes analysis of a long time series of phytoplankton cell counts collected alongside environmental variables in the Cariaco Basin time series site, as well as larger climatic indices. The work recapitulates early analysis of trends in phytoplankton dynamics associated with environmental variables; previous work determined phytoplankton classes according to HPLC pigment concentrations and examined environmental data from water samples collected at the CARIACO Ocean Time Series site. The current study built upon this work and analyzed genus-level phytoplankton data using cell counts and incorporated climate indices (AMO and ENSO signals). The present work uses gradient forest analysis to identify thresholds in environmental variables to better understand drivers of phytoplankton community diversity and abundances. While there are some limitations of this approach (i.e., with monthly data, there is the potential of time-aliasing and a lack of resolution with which to identify thresholds), the long-time series helps to overcome some of the limitations in ability to detect changes that correspond to environmental thresholds. There are few long-term data sets of this type, which make the study valuable for exploring relationships between environmental factors and biological production over time. At a time when climate change is threatening ecosystems worldwide, such data and their analyses are critical to understand and predict the impacts of climate shifts.
\end{generalcomment}
\begin{revresponse}[Thank you for your feedback.]

\end{revresponse}

\begin{generalcomment} (continued) 
	The paper is generally well-written and logically organized and I appreciated the presentation of data, particularly the way that missing data were represented in Figure 2. Although I have a few suggestions to improve a couple of the figures, they were generally easy to understand and interpret.
\end{generalcomment}
\begin{revresponse}[Thank you for your feedback.]
\end{revresponse}

\begin{revcomment}
	There are several spots that could benefit from tighter synthesis of ideas and greater clarity in presentation, mainly in the Results section, where findings were sometimes presented as stand-alone statements rather than woven together into a cogent story or line of thinking. The authors did a nice job articulating how the work built upon previous analyses and explained some of the limitations of the work; however, I do think they missed an opportunity to discuss how gaps in data in the last few years of the time series might affect their conclusions. The biggest question I have about this study is how missing data affect the conclusions drawn by the authors. How were missing data in 2004, 2014, 2015, 2016, and 2017 addressed in the cluster analyses? The crux of the work lies in the fact that there is a perceived oscillation in phytoplankton communities between two thresholds (2004 and 2014). Yet, there are only 3 years post-2014, and two of them are characterized by large data gaps (in 2015 there are $>$ 4 months of missing data, while in 2014 there are $>$ 2 but $<$ 4 months of missing data). No data from 2016 or 2017 are shown at all in Figure 2. The authors did not discuss this in the section on limitations in the Discussion, which I found surprising; how susceptible to these missing data are their conclusions? I think it would help to identify the particular months of missing data in year where there were significant gaps; otherwise, averaging values when seasonal highs or lows are absent risks skewing the results, which the authors note in the text in at least one section. Aside from a brief comment about missing chlorophyll data in 2013 (lines 351-352), there wasn’t much discussion about the importance of missing data.
\end{revcomment}
\begin{revresponse}
    We thank the reviewer for raising this important point. We have expanded our discussion of missing data and added a dedicated subsection (Section~2.4.3) clarifying how different analyses handle data gaps.
    
    For annually aggregated analyses (z-scores in Figure~\ref{fig:zscore}, hierarchical clustering in Figure~\ref{fig:clustering}a, species turnover in Figure~\ref{fig:clustering}b, and annual means in Figure~\ref{fig:divts}), we excluded years with fewer than six monthly observations (1995, 2016, and 2017). The hierarchical clustering uses presence-absence data, which is relatively robust to missing months because a genus need only be detected once within a year to be recorded as present. Monthly-resolution analyses (NMDS ordination, density distributions, gradient forest) include all available observations. We have now additionally conducted sensitivity analyses using `core months' consistently sampled across 2014-2017 to verify that conclusions were not artifacts of uneven sampling.
    
    Regarding the post-2014 period: although 2014 and 2015 have gaps, both years retain observations from both upwelling and rainy seasons, reducing the risk of seasonal bias. To further assess the robustness of our conclusions, we compared monthly observations across three periods (Early Cluster~1: 1996--2004; Cluster~2: 2005--2013; Late Cluster~1: 2014--2017) using Wilcoxon rank-sum tests (Table~\ref{tab:sup:period_comparison}). For key variables including SST, salinity, chlorophyll~$a$, and several functional groups, Early~C1 and Cluster~2 differ significantly, and Cluster~2 and Late~C1 differ significantly, while Early~C1 and Late~C1 do not---a pattern consistent with partial recovery to pre-2005 conditions. Additionally, independent sampling near Margarita Island in the Cariaco Basin \citep{gomez_gaspar_variacion_2025} shows a return to elevated surface chlorophyll during 2014--2017, corroborating our interpretation.
    
    We have added discussion of these points to Section~4.3 (Limitations).
\end{revresponse}

\begin{revcomment} 
	In a similar vein, I found it a little hard to follow how they were grouping the data in some sections of the manuscript. In some analysis, they aggregated data by year (including 12 months of data into a yearly average), while in other analyses, report seasonal values (e.g., upwelling vs. downwelling seasons). But even monthly-averaged data is subject to time-aliasing; they might consider a discussion on the implications and impacts of time averaging. The same can be said for depth-averaging; I would have liked to have seen data on mixed layer depth and solar insolation to better understand the implications of data interpolation or the conclusions that the chlorophyll maximum was deeper in some years than others.
\end{revcomment}

\begin{revresponse}
    We have added a paragraph to Section~2.4 (line~\lineref{R:ClarifyAggregation}) that clearly summarizes which analyses use annual aggregation versus monthly resolution, and which use depth integration over 0--100~m versus the four depth intervals shown in Figure~\ref{fig:divts}. We hope this clarifies the data grouping throughout the manuscript.
    
    Regarding time averaging: we note that our in-situ data represent discrete monthly samples, not averages derived from higher-frequency measurements. Only the climate reanalysis data (ERA5) represent monthly averages. Annual aggregation was applied selectively to emphasize interannual trends, while monthly-resolution analyses (NMDS, gradient forest) preserve seasonal signals. Figure~\ref{fig:divts} was included specifically to show both depth-resolved and seasonal patterns, complementing the depth-integrated analyses elsewhere.
    
    Regarding mixed layer depth and chlorophyll maximum dynamics: our analysis focuses on depth-integrated properties to capture large-scale community changes, rather than on vertical structure per se. Mixed layer depth in the Cariaco Basin is consistently shallow (typically $<$35~m; \cite{muller-karger_scientific_2019}) and the 21$^{\circ}$C isotherm depth is more commonly used as a proxy for upwelling intensity in this system. Our references to a deeper chlorophyll maximum during stratified periods are based on previous research \citep{lorenzoni_characterization_2015, pinckney_phytoplankton_2015}, which documented euphotic zone depths of 60--75~m during low-production periods compared to 15--30~m during upwelling. Our depth-resolved data in Figure~\ref{fig:divts} are consistent with these earlier findings.
\end{revresponse}

\begin{revcomment} 
	Overall, I think the work merits publication; I would just like to see more careful discussion of the limitations imposed by gaps in data as well as implications of broad grouping (i.e., ‘nanoflagellates’) vs. genus-level data on diversity metrics, and some work to synthesize and clarify sections of the text that feel a bit confusing to the reader.
\end{revcomment}
\begin{revresponse}
    We thank the reviewer for these suggestions. As detailed in our responses above, we have added discussion of data gaps and their implications to Section~2.4.3 and Section~4.3, and clarified the data aggregation approaches in Section~2.4.
    
    Regarding the `nanoflagellates' grouping: this broad category appears only in the z-score analysis (Figure~\ref{fig:zscore}), where it illustrates interannual trends in functional group biomass. Nanoflagellate counts were reported as such in the original data and were not identified to finer taxonomic levels. Importantly, nanoflagellates were excluded from all diversity metric calculations, which are based exclusively on genus-level identifications (as noted in lines 192--193 and 272--273). We have added clarifying text to ensure this distinction is clear to readers.
    
    We have also revised sections of the Results to improve synthesis and flow, as suggested.
\end{revresponse}


\begin{revcomment}
	Abstract
    
    The statement about phytoplankton diversity could be better articulated here; it isn’t clear what the implications of a lack of change in diversity over a part of the data set are. Perhaps just add a clause indicating what this signifies. Otherwise, it feels like a result out of place.
\end{revcomment}
\begin{revresponse}
	We have reformulated that sentence to highlight the finding, that community evenness was higher during cluster 2. Indeed, we have no clear hypothesis or result to explain the lack of change in diversity, however the change in evenness is in line with the common understanding of the link between upwelling, productivity and community evenness.
\end{revresponse}

\begin{revcomment}
	Abstract (continued)
    
    This sentence would benefit from a bit of elaboration: “…found the AMO index was the strongest predictor” – please indicate what the predictor is.
\end{revcomment}
\begin{revresponse}
	We have reformulated the sentence to clearly state that it is a predictor of changes in the phytoplankton community composition.
\end{revresponse}

\begin{revcomment}
	Abstract (continued)
    
    The conclusion that the pattern is cyclic depends on whether the data from 2014 onward are representative of the full calendar year (despite missing data); the relationship with the AMO is interesting regardless of cyclicity
\end{revcomment}
\begin{revresponse}
	NOTE: Do some tests with the data and then come back to highlight the reasons why the cyclic pattern seems robust.
\end{revresponse}


\begin{revcomment}
	Introduction
    
    It would be great to include a little more explanation about the AMO and AMO anomaly to better understand its impact on water column characteristics and phytoplankton populations; it ends up being a major part of the story.
\end{revcomment}
\begin{revresponse}
	We have included a clearer discussion of the current knowledge of the effect of AMO phases on the Caribbean Sea and the Cariaco Basin to clarify this.

    \begin{changes}
        The oscillations of the AMO index are strongly correlated with temperature indices in the Caribbean Sea, where a negative phase is generally correlated with warmer and drier conditions and a positive phase with colder and wetter conditions (Stephenson et al. 2014). A negative AMO phase has also been linked to a southward shift in the ITCZ (Knight et al. 2006). Such a southward shift of the ITCZ is correlated with increased wind-driven upwelling in the Cariaco Basin (Taylor et al. 2012). 
    \end{changes}
\end{revresponse}


\begin{revcomment}
	Introduction (continued)
    
    The authors state that “Data from 2014 to 2017, the final years of the CARIACO (Carbon Retention In A Colored Ocean) timeseries, indicated a return to strong upwelling intensities and high numbers of diatoms and dinoflagellates (Muller-Karger, Astor, Benitez-Nelson, Buck, et al., 2019).” It would be nice to specifically state the hypothesis that larger climatic factors influence the shift to stronger upwelling and thus, increased numbers of diatoms and dinoflagellates; it would also be great to indicate that the former data came from HPLC pigment analysis so that the reader is clear on how the present work is different from that already done. Examining the data at the class level (i.e., through pigment analysis by HPLC) vs. at the genus level could produce a different understanding of responses to environmental variables.
    
\end{revcomment}
\begin{revresponse}
	We have reformulated that section and included both the hypothesis and specific references to which data is the foundation for Muller-Karger et al.'s suggestion, that there was a return to strong upwelling intensities and the accompanying phytoplankton community. Please note, that this is based on aggregated microscopy cell counts of Diatoms and Dinoflagellates, and not HPLC data.
    The present work explores the cell count data in much greater detail than the CARIACO time series review paper, and with a different methodology.
    We have also added a sentence to clearly state our hypothesis that these changes were driven by large-scale climatic factors, thank you for pointing this out.
\end{revresponse}

\begin{revcomment}
	Introduction (continued)
    
    Line 119: Please state what observed changes are being discussed here; it would help orient the reader to the purpose of this paragraph
\end{revcomment}
\begin{revresponse}
	We have added the specific trends that Taylor et al. 2012 is presenting, which hopefully clarifies this.
    In general, we aim to highlight the novel aspect of our study, that we analyze the trends not as unidirectional, but include more data and find the cyclic nature of it.
\end{revresponse}


\begin{revcomment}
	Methods

    Section 2.3 would benefit from some re-working. The authors explain that they examined meteorological data (wind, precipitation, evaporation) as well as climate indices, but do not explain in what ways the data were used. The second paragraph in this section introduces the idea of exploring two climatic indices, but then only describes one, with the second presented in the following paragraph. These could be consolidated for greater clarity.
\end{revcomment}
\begin{revresponse}
	We have consolidated the last two paragraphs of this section, and add a sentence to clarify for which part of our analysis these values were used.
\end{revresponse}

\begin{revcomment}
	Methods (continued)

    It would be helpful to know which months constituted missing data; there are missing data and substantial gaps for each of the years 2014, 2015, 2016, and 2017, as well as 2004 (for phytoplankton data). How important were the missing data in 2004 or these latter years in driving a perceived shift in species distribution/diversity?
\end{revcomment}
\begin{revresponse}
	Thanks, we ...
\end{revresponse}

\begin{revcomment}
	Methods (continued)

    Lines 195 – 198: The authors refer to easterly winds in referencing zonal winds, but then say they extracted the eastward component of the wind – easterly vs. eastward generally refer to different directions (easterly = from the east, eastward = toward the east). The authors should clarify which is correct and then adjust the language to use similar terms for wind direction. It is also a bit unclear whether they only used the easterly/eastward component of the 10 m winds, whether they used the absolute value of wind speed at 10 m height, or whether they accounted for wind direction in their analyses.
\end{revcomment}
\begin{revresponse}
	We used the negative u10 value in our analysis, which corresponds to easterly winds. We have edited the sentences to include this information.
    Since wind-driven upwelling is primarily driven by these easterly winds, we did not consider any other wind direction or data in this part of the analysis, it is simply the negative u10 value.
\end{revresponse}


\begin{revcomment}
	Methods (continued)

    Lines 211 – 215: Incomplete sentence.
\end{revcomment}
\begin{revresponse}
	We have improved the grammar and wording in this section of the paragraph.
\end{revresponse}

\begin{revcomment}
	Methods (continued)

    Section 2.3.1: To clarify the assumptions that are made in calculating depth profiles, it would be good to indicate the depth frequency of interpolated values – ie, depth integration was achieved by summing interpolated values at what depth intervals? In addition, it would be nice to report or show the mixed layer depth in relation to the selected depth intervals/bins to better understand influences on subsurface chl maxima. If PAR data were not collected, the data aren’t available to discuss; however, it would be good to acknowledge the influence that the euphotic zone depth has on the depth distribution of phytoplankton.
\end{revcomment}
\begin{revresponse}
	We have added the discrete depth intervals that data from the CARIACO time series, and in particular the variables we used, were sampled at.
    
    The mixed layer is generally not a reliable measure in the tropcial coastal ecosystem of the Cariaco Basin. Please see the 2019 Review of Muller-Karger et al. for a plot of mixed layer depth across the time series.
    In other studies, the 21°C Isotherm was generally used as a proxy for upwelling strength, instead of the mixed layer depth. 
    NOTE: I need to somehow address the euphotic depth questions, I can check the MLD data again, and compare averages between clusters and add interannual plots. Mention that there is no PAR and only euphotic depth data until 2012.
\end{revresponse}

\begin{revcomment}
	Methods (continued)

    Section 2.4 – Data analysis:
    The authors note that data from 1995, 2016, and 2017 were excluded due to missing data, with more than 6 months of data missing (i.e., 50\% of each year). The missing data make it difficult to be confident in the cyclical nature of change reported in the manuscript. Were there also missing data for phytoplankton cell counts in these years? In Lines 334-335, the authors note that “Although data are sparse, nutrients and biological variables show a strong increase from 2014”
\end{revcomment}
\begin{revresponse}
	NOTE: Should I mention explicitly in the text that 2015 and 2017 only include a single sampling? this seems to have confused the reviewer.
    For these years there clearly is also missing phytoplankton community data, but this is irrelevant for everything except the community clustering.
    
    We still have data for the years 2014 and 2015 and already answered to another question about the supporting information that points towards a return to a stronger upwelling regime.

    NOTE: I should probably simply expand on the sentence that is quoted here, and explicitly state which data is missing, and why we still think there is a cyclic pattern!
\end{revresponse}

\begin{revcomment}
	Methods (continued)

    Lines 241-243: This topic sentence would benefit from re-working to improve clarity. Is it correct that the authors took all cell counts from the different depths sampled for a given month, created interpolated values, averaged them over depth to get a single representative value for a taxon for that month, did this for each month, and then created an average that included all months of the year? The way it is phrased leaves a little uncertainty on exactly where the averaging happened.
\end{revcomment}
\begin{revresponse}
	We have reworded this sentence to clearly state that the averaging happens after integrated monthly cell counts at the species and genus level are summed up across functional groups. The averaging happens with the total of cell counts per functional group for each month.
\end{revresponse}


\begin{revcomment}
	Methods (continued)

    Section 2.4.2
    Please cite the software used to carry out gradient forest analysis (R package, Gradient Forest?)
\end{revcomment}
\begin{revresponse}
	We have modified the paragraph to more clearly identify the package used and added the appropriate citation.
\end{revresponse}


\begin{revcomment}
	Methods (continued)

    It would be helpful if the authors explained why they used integrated cell count data for the gradient forest analysis, but interpolated nutrient data.
\end{revcomment}
\begin{revresponse}
	Note: Good point, why did I do this?? Should I redo with integrated nutrient data?
\end{revresponse}



\begin{revcomment}
	Results

    Some sections in the results are quite descriptive with limited synthesis, making it read like a list of observations rather than parts of the story.
\end{revcomment}
\begin{revresponse}
	We went through the results section again, and added sentences to highlight the cohesiveness and parts of the result.

    NOTE:TODO this!
\end{revresponse}

\begin{revcomment}
	Results (continued)

    Figure 2: I think the figure would be more effective if the color spectrum was reversed; usually, the convention is that high values are red tones, while low values are blues – the first time I looked at this plot, I assumed that the reds were positive anomalies rather than negative ones.
\end{revcomment}
\begin{revresponse}
	We appreciate your input, and did consider this coloring carefully. We did specifically choose the blue color to represent the "colder" stronger upwelling conditions if cluster 1, and red colors representing warmer, stratified conditions of cluster 2.
    Note: Create a reversed Color version of this plot, since the explanation does not really make sense!
\end{revresponse}


\begin{revcomment}
	Results (continued)

    Line 325: The authors refer to ‘periods’ of stronger upwelling, but they are discussing annual data (i.e., average of monthly values) – so, the unit of consideration would be ‘years’ of strong upwelling/anomalies, based on the data being discussed. If they mean to say that nitrate/phosphate tend to be elevated under periods of stronger upwelling (i.e., event-scale), it would be helpful to specify this – that is, that strong upwelling periods in a given year may result in positive nitrate/phosphate anomalies in the yearly average.
\end{revcomment}
\begin{revresponse}
	We have modified the wording in that sentence to more clearly state that years with stronger upwelling periods result in elevated annual mean anomalies.

    NOTE:This sentence clearly needs a rewording!
    
\end{revresponse}

\begin{revcomment}
	Results (continued)

    Lines 337-341: The last part of this paragraph would benefit from additional clarity.
\end{revcomment}
\begin{revresponse}
	We have reworded the last part of the paragraph to imrpove clarity.

    Note:CHECK THIS AGAIN!
\end{revresponse}

\begin{revcomment}
	Results (continued)

    In the paragraph discussing genus diversity (lines 360 – 372), the idea that the transition from high to low to high diversity index values represents a ‘recovery’ could use some justification; this assumes that low diversity represents an impairment, and the change back to a higher diversity regime represents a return to better conditions. The last two lines of this paragraph could be improved for clarity; lines 369 – 371 are a little confusing. The middle of the paragraph discusses annual values, but this line calls out diversity “throughout 2000”, with the first and last parts of the paragraph discussing differences between upwelling and downwelling periods. Overall, it was a little hard to follow the discussion about intra- vs. inter-annual patterns as presented. It would help if the purpose of the analyses was described first – i.e., to distinguish between upwelling vs. downwelling seasons in terms of genus diversity – and then place these observations into the larger context of inter-annual variability.
\end{revcomment}
\begin{revresponse}
	We have reformulated the paragraph.
\end{revresponse}


\begin{revcomment}\label{comment:clustering-dominance}
	Results (continued)

    Section 3.2:
    Line 395 – If the data were divided into 2 clusters, why did the authors highlight the “later part” of cluster 1? Given the fact that the first cluster was dominated by Thalassiosira, which transitioned to Emiliania (note that Emiliania is misspelled in the manuscript) and the second cluster is dominated by Pseudo-nitzschia followed by Emiliania, it would seem that what they are describing is less of a distinct 2-cluster situation and more of a transition.
\end{revcomment}
\begin{revresponse}
	Here we are looking at two separate ways of interpreting the data. The clustering was performed using the binary Jaccard distance matrix of yearly aggregated cell counts. The binary data here of course does not take into account the community structure (dominance).
    We wanted to include the dominant taxa to highlight this aspect of the community data, which of course does not directly correspond to clustering from presence/absence data. 
\end{revresponse}


\begin{revcomment}
	Discussion

    Line 456 – Do the authors mean “confounding” or corresponding effects? The context seems better suited for “corresponding”
\end{revcomment}
\begin{revresponse}
	We have corrected the phrase to use the word "corresponding", since this is what we meant to say. 
\end{revresponse}


\begin{revcomment}
	Discussion (continued)

    Line 460 – It would be helpful to put the reduction in cell counts into context – 5 times lower? 10 times?
\end{revcomment}
\begin{revresponse}
	NOTE: Still have to check this with the data!
\end{revresponse}


\begin{revcomment}
	Discussion (continued)

    Lines 468-470 – Did the phytoplankton community return to previous composition after 2014? As stated above, the dominant genera from 2014 – 2017 included Emiliania, Thalassionema, and Thalassiosira (present throughout the time series); it seems as though the dominant species showed more of a transition than stepped differences, with the early period having Pseudo-nitzschia and Thalassiosira as dominants, transitioning to Emiliania and Thalassiosira. If the rest of the community is what is driving the conclusion that taxonomic shifts between the clusters, it would be nice to elaborate a bit on this here.
\end{revcomment}
\begin{revresponse}
	Similar to our answer to \autoref{comment:clustering-dominance}, this is based on the different approach of using binary presence/absence data for the yearly clustering, so dominance of genera is not relevant here.
\end{revresponse}


\begin{revcomment}
	Discussion (continued)

    Line 476 - What is the size limit for detection of cells in this analysis? Depending on the magnification used, cells as small as 2-3 um could be detected. Some diatoms are very small, so it may not be the case that a shift to smaller cells did not include diatoms (line 477; I would suggest specifying ‘large’ diatoms here).
\end{revcomment}
\begin{revresponse}
	The magnification used in identifying cells was 100x according to the Methods Handbook of the CARIACO time series. We stated in our discussion, that only microphytoplankton (cells > 20 µm) were identified. To make this clearer in our manuscript, we have added a sentence to the methods section to mention the magnification and size limit.
    
    That small diatoms may have gone undetected is a valid point and we have added the specification of "large" diatoms in the sentence. 
\end{revresponse}


\begin{revcomment}
	Discussion (continued)

    Line 498: typo – ‘groupings’, not ‘groupsing’
\end{revcomment}
\begin{revresponse}
	We have fixed the typo.
\end{revresponse}



\begin{revcomment}
	Discussion (continued)

    Line 499 – It would be helpful to know the mixed layer depth to understand the significance of 55 m depth
\end{revcomment}
\begin{revresponse}
	We have added another reference to Pinckney et al. 2015 directly behind this sentence, since this is where we take this claim from.

    Generally, the MLD is a less reliable estimate of mixing in the tropical coastal ecosystem of the Cariaco Basin. It averages around 22 meters depth across the time series and varies between 8 and 45 m at the extremes (Muller-Karger et al. 2019). 

    NOTE:From Muller Karger REVIEW 2019:
    "The euphotic-zone depth (1\% of surface irradiance) during low-production periods reached 60–75 m, but it was 15–30 m during upwelling due to biomass accumulation in surface waters (Lorenzoni et al. 2011, Pinckney et al. 2015)."
\end{revresponse}

\begin{revcomment}
	Discussion (continued)

    Line 509 – When nanoflagellates are present at increased numbers in the community, this would influence richness estimates, since the various genera encompassed by this polyphyletic group are not necessarily captured accurately; it would be helpful for the authors to discuss how a broad grouping might influence estimates of genus richness and thus the clusters and patterns observed in the data. Lines 512 – 516 would benefit from consideration of the effect of a grouping of nanoflagellate taxa on evenness metrics.
\end{revcomment}
\begin{revresponse}
	As stated in our methods, we did include the nanoflagellate counts in our z-score analysis, but excluded all counts that are not at the genus level from all other parts of our analysis.
    Hence, there is no influence of nanoflagellate counts on any of our calculated diversity metrics.
\end{revresponse}


\begin{revcomment}
	Discussion (continued)

    Line 580 – Is there evidence of diatom blooms, or are the authors meaning to say higher diatom cell counts?
\end{revcomment}
\begin{revresponse}
	We have reformulated the sentence to indicate that we are talking about growth in cell counts and are not specifically talking about diatom blooms. 
\end{revresponse}


\begin{revcomment}
	Discussion (continued)

    Line 625 – missing reference (denoted by ?, ?)
\end{revcomment}
\begin{revresponse}
	We have corrected the reference.
\end{revresponse}


\begin{concludingresponse}[]
	Thank you for your valuable comments on our manuscript. 
\end{concludingresponse}



% \begin{revcomment}\label{comment:work-not-good}
% 	The work is not really good.
% \end{revcomment}
% \begin{revresponse}
% 	:(
% \end{revresponse}

% \begin{revcomment}
% 	You forgot to cite a very important reference (where I am an author)!
% \end{revcomment}
% \begin{revresponse}
% 	We are aware that citations on Google Scholar are very important to you.
% 	Therefore, we added reference \cite{ReviewerReference}.
	
% 	Also check out our article \cite{Besser2020}.
	
% 	\printpartbibliography{ReviewerReference,Besser2020}
	
% 	And btw, your \autoref{comment:work-not-good} was mean! (We can use the \verb|\autoref| command.)
% \end{revresponse}


% \reviewer
% \begin{revcomment}
% 	Did you know, that the references can be separated for the individual reviewers?
% \end{revcomment}
% \begin{revresponse}
% 	Yes. When using \href{https://www.ctan.org/pkg/biblatex}{biblatex}, you can use the \texttt{refsection=section} option to achieve that.
% 	If we cite a new reference like \cite{Besser2021} here, it will again be number [1].
	
% 	Note that you might have to run \texttt{pdflatex} and \texttt{biber} multiple times.
	
% 	And reference [1] for \autoref{rev:2}~\cite{ReviewerReference} is now number [2].
	
% 	\printpartbibliography{Besser2021,ReviewerReference}
% \end{revresponse}



\end{document}