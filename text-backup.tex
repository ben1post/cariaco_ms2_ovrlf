










%%%%%%%%%%%%%%%%%%%%%%%%%%%%%%%%%%%%%%%%% BELOW HERE ARE INTRO DRAFT TEXTS! %%%%%%%%%%%%%%%%%%%%%%%%%%%%%%%%%%%%%%%  
    
    % "Changes in plankton biodiversity in response to regional changes in oceangraphic conditions" driven by climate oscillaitons (from REVIEW 2019).
    %% Changes in Functional Groups

    %% Changes in Diversity



    % Now we come into the picture: Which gap are we adressing?

    % FOCUS ON Community Changes + BOTTOM UP


    
    These changes are believed to be influenced by various factors, including climate variability, nutrient fluxes, and ocean circulation patterns.
    % WHAT ARE THE DRIVERS OF COMMUNITY CHANGE? Bottom up OR top down, we focus on bottom up...

    Describe physical changes!

    - AMO (ref to 2019 Review)
    As the CARIACO time series was implemented in 1995, we now know that the Atlantic Multidecal Oscillation (AMO) was shifting to a positive phase for the next 20 years, which resulted in the largest oceanographic anomalies of the last century (McCarthy et al. 2015). These climatic changes had a profound impact throughout the Atlantic (Alexander et al. 2014, Nye et al. 2014) and in the Cariaco basin \cite{taylor_ecosystem_2012,pinckney_phytoplankton_2015}.
    - ENSO teleconnection (ref to 2019 Review)
    The Caribbean experiences the most marked of the El Ni  ̃no–Southern Oscillation teleconnections in the Atlantic Ocean, with warmer sea surface temperatures and trade winds blowing more to the north-northwest during the El Ni  ̃no–Southern Oscillation (Enfield & Mayer 1997).

     - Understanding the dynamics of the Cariaco Basin is crucial to understanding the broader ecological changes in tropical marine ecosystems. + Carbon fluxes in general!


    % Finally say what WE are going to do! What are the open questions, what is new?

    
    %% try to identify drivers of shift in community with Gradient Forest






    %%%%%%%%%%%%%%%%%%%%%%%%%%%%%%%%%%%%%%%%%%%%%%%%%%%%%%%%%%%%%%%%%%%%%%%%%%%%%%%%%%%%%%%%%%%%%%%%%%%%%%%%


    
    %%% Instead of going paper by paper below, I should go topic by topic (i.e. all points made on "DIVERSITY" should be in a paragraph, and all on changes in FGs or Size Classes in another etc...)
    
    Since the inception of the time series, the Cariaco basin has undergone marked shifts...

    The time series includes multiple measurements of the phytoplankton community, which are the bulk fluorometric chlorophyll, microscopy cell counts, and high-performance liquid chromatography (HPLC) pigment analysis. Of these, all cover the entire time series except for the HPLC pigment data, which was performed by three different labs and with a gap between 2000 and 2006 \cite{muller-karger_scientific_2019}. 

    The most often measured indicator of bulk phytoplankton biomass is fluorometric Chlorophyll \textit{a} concentration, which gives a general measure of biomass standing stock and distribution in the water column. A more advanced methodology, high-performance liquid chromatography (HPLC), allows individual pigment types to be quantified. These pigment distributions can be used to estimate the groups of phytoplankton that contribute to biomass. The most specific methodology to quantify phytoplankton communities is the counting and identification of individual phytoplankton cells, either manually under a microscope or automated in flow-through systems. The major drawbacks are the size limit of resolving smaller cells and, if done manually, the complexity of identifying phytoplankton species. 


    % PAPER TIMELINE:
    - Taylor 2012 ()
    \citeA{taylor_ecosystem_2012} analyzed data up to 2010 and found significant decadal trends of increase in sea surface temperature (SST), coinciding with a decrease in upwelling intensity and phytoplankton bloom intensities, which the authors related to the impact of global warming. The phytoplankton community identified via microscopy showed a shift towards smaller taxa after 2004, which coincided with a collapse in commercial sardine landings and an increase in mesozooplankton biomass. The authors explained the change in physical regime in the Cariaco basin with a north-easterly shift of the Inter-Tropical Convergence Zone (ITCZ) and a resulting weakening trend in Trade Winds.
    
    - Mutshinda 2013 ()
    Using a hierarchical Bayesian modeling approach, changes in dominant species quantified via microscopy cell counts between 1995 and 2011 were found to be best predicted by changes in temperature, pH and irradiance, and less so by salinity or macronutrient concentrations \cite{mutshinda_environmental_2013}. In particular, higher temperatures were associated with a lower abundance of diatoms. 
    
    - Irwin 2015 ()
    With the same dataset of microscopy cell counts and a MaxEnt statistical modeling approach, \citeA{irwin_phytoplankton_2015} calculated realized niches for the dominant species for SST, irradiance and nitrate concentration. As trends in the realized niches were observed over the course of the time series, this seemed to show the adaption of species to climate change effects. 
    
    - Pinckney 2015 ()
    The phytoplankton community was investigated in detail by \citeA{pinckney_phytoplankton_2015}, using the microscopy counts and HPLC data from the time ranges 1996 to 2000 and 2006 to 2010. Between these two time periods, they observed that the drastic reduction in diatoms counted using microscopy did not correspond to a similar reduction in diatom abundance calculated from the pigment composition, which is most likely explained by a shift to diatom species with smaller cell sizes. Additionally, the pigment data indicates a deepening of the chlorophyll maximum, an increase in diversity and a reduction in seasonal variability in the community composition. These changes are most likely a response to the reduced upwelling  and a deepening of the euphotic zone associated with a decrease in the strength of seasonal trade winds. Interestingly, the shifts in the phytoplankton community did not result in a significant change to the export of particulate organic carbon (POC) to depth as measured by sediment traps deployed during the time series. 

    - Lorenzoni 2015 ()
    \citeA{lorenzoni_characterization_2015} focused their analysis on HPLC data retrieved between 2006 and 2012 and related seasonal changes in the pigment composition to spectral absorption from remote sensing data and concentrations of organic carbon. They observed a high seasonal variability, with microphytoplankton (mostly diatoms) dominating upwelling periods and picophytoplankton (mostly cyanobacteria) dominating during the rainy season. These community compositions were also reflected in the POC:TChla ratio, which was higher for diatom dominated assemblages and lower for those dominated by cyanobacteria. 
    % "The difference between the taxonomic data and the pigment data suggests that the microscopy taxonomy may not capture the variability in the pico-phytoplankton. Since the beginning of the time-series in 1995, there has been a change in the abundance of phytoplankton size classes based on diagnostic pigments: micro-phytoplankton has decreased year-round, while picophytoplankton has increased, particularly during the rainy season. This is consistent with the decrease in larger phytoplankton previously reported for the region." Lorenzoni et al. 2015
    

    
    - Finally, add the info from Review about Phytoplankton community, about community and upwelling returning in 2014 to previous levels/state.
    - Understanding the dynamics of the Cariaco Basin is crucial to understanding the broader ecological changes in tropical marine ecosystems. + Carbon fluxes in general!

    % Hone in on Diversity Aspects, and what has been done there?
    - Highlight diversity aspect of Pinckney
    - Highlight Mutshinda etc. How they used Niches/Phyto data
    - What are the questions in Diversity TS data? -> Esteban's papers
   

    
    % In recent years, significant changes have been observed in phytoplankton populations within the Cariaco Basin. These changes are believed to be influenced by various factors, including climate variability, nutrient fluxes, and ocean circulation patterns. Understanding the drivers and impacts of these changes in the phytoplankton community is crucial to understand the broader implications for marine food webs and carbon cycling in this and similar ecosystems.
    
    "The Cariaco Basin has undergone marked shifts..." Talk about ecosystem, what has changed, reference prev literature

        % below is from 2019 REVIEW
        %Hydrographic observations (Figure 2) show strong seasonality and interannual variability in the upper 400 m in the Cariaco Basin. This is controlled by seasonal and interannual changes in the wind and in the intensity of the geostrophic Caribbean Current (Muller-Karger et al. 1989). Primary upwelling was strongest between November and May. In general, during upwelling, sea surface temperatures ranged between approximately 23\degree C and 25\degree C. The secondary upwelling that occurs between June and August is shorter ($\sim$ 5 weeks) and is approximately 1.5\degree C warmer than the primary upwelling. It is driven by Ekman transport and an intensification of the curl of the wind close to the southern coast of the Caribbean (Rueda-Roa \& Muller-Karger 2013, Rueda-Roa et al. 2018). Upwelled waters move offshore in plumes that extended as far as 50 km to more than 250 km from the coast (Muller-Karger et al. 1989, 2010). From July to October, there is strong stratification in the upper 50 m of the water column after the wind weakens and the rain and rivers deliver less saline waters (Figure 2).
        %Sea surface temperatures ranged from approximately 28\degree C to 30\degree C during the period of stratification (Figure 2). The mixed-layer depth varied from less than 10 m to 35 m during the 21-year time series (Figure 3). Satellite sea surface temperature and ocean color observations show that upwelling pulsates throughout the year at the scale of days to weeks, leading to marked variation in primary production (Figure 3). CARIACO rigorously characterized the annual seasonal variation in primary productivity and stratification. The length of the time series was key in capturing interannual variations that spanned decades. Satellite sea surface temperature and field data show that upwelling was stronger between 1996 and 2002. After 2002, trade wind intensity became weaker (Taylor et al. 2012).
        The field observations, particularly those in the upper 50 m, showed what appeared to be a return to stronger upwelling conditions starting in 2014 (Figure 2a).


        %“The Caribbean experiences the most marked of the El Nino–Southern Oscillation telecon- ˜ nections in the Atlantic Ocean, with warmer sea surface temperatures and trade winds blowing more to the north-northwest during the El Nino–Southern Oscillation (Enfield & Mayer 1997). ˜” ([Muller-Karger et al., 2019, p. 4](zotero://select/library/items/7YW7MFTU)) ([pdf](zotero://open-pdf/library/items/TTZQBAXW?page=4&annotation=D3H7YG7Y)) link to El Nino of Cariaco/Caribbean sea!

        %“When the CARIACO time series was implemented in 1995, major changes were happening in the Atlantic. We now know that at that time, the Atlantic Multidecadal Oscillation (Kerr 2000, Enfield et al. 2001) started shifting to a positive phase, which then persisted for more than 20 years and led to oceanographic anomalies in the Atlantic that were larger than any recorded over the previous 100 years (McCarthy et al. 2015). The changes had profound biological impacts throughout the Atlantic (Alexander et al. 2014, Nye et al. 2014), including in the Cariaco Basin (Taylor et al. 2012, Pinckney et al. 2015).” (Muller-Karger et al., 2019, p. 3) Interesting aspect that the AMO changed, as apparently the main climate indices to affect the Atlantic/CARIACO
        

    - Why community structure is important, taxonomic diversity (aspects of diversity) - why look at taxonomic diversity?


    - some general intro to DIVERSITY.. what is it? Why important?
    

    % "Phytoplankton is the major primary producer group in marine ecosystems and at the same time a highly suitable indicator for ecosystem change. Due to its high diversity and short generation time, phytoplankton responds quickly to changes in the environment (Reynolds 1998). Especially changes in temperature and nutrient supply are known to change phytoplankton composition and biodiversity, with higher temperatures and lower nutrients favouring small species (Litchman et al. 2007; Winder and Sommer 2012), whereas Si:N and N:P supply ratios change the relative importance of diatoms and other groups such as dino(flagellates) and cyanobacteria (Sommer et al. 2004; Makareviciute-Fichtner et al. 2020)." from Di Cavalho et al. 2023
    

    %"Whereas the positive biomass response to high nutrient and light availability is well known (Tilman 1982; Leibold 1999; Siegel et al. 2013), the biodiversity response is much more complex. First, biodiversity itself is a multifaceted construct, comprising the number of different taxa (e.g. species richness) and the equality of species contribution to the assemblage as aspects of local alpha diversity (evenness). But the change in richness is only the net effect of immigration and local extinction, while the turnover in species composition with time is an additional aspect of biodiversity. Extending to larger spatial scales, heterogeneity in space (beta diversity) and composition of the regional species pool (gamma diversity) are further biodiversity dimensions. Second, for each of these facets, all metrics are effort- and scaledependent. Thus, in contrast to other aspects of ecosystem assessments, biodiversity metrics do not qualify to set absolute target values for good or bad status and effective ecosystem management. But, the relevance of the metrics arises from analysing their temporal trends over a long time period. Third, no single variable captures even the most important aspects of community composition and change. Therefore, the analysis of phytoplankton biodiversity requires a multimetric approach." from Di Cavalho et al. 2023 (with Hillebrand)
    
    - why are we looking at Functional Groups?
    %"The complexity of biodiversity assessment has led to the development of other approaches addressing compositional change in plankton communities, most prominently the creation of functional groups. Functional groups are meant to comprise species that show similar responses to environmental drivers and share morphological and/or physiological characteristics. Functional group indicators have been shown to be relevant for describing community structure and biodiversity and are more comparable with other studies than species-based indicators (Mouillot et al. 2006)." from Di Cavalho et al. 2023 (with Hillebrand)
    
    %% Overarching theme: Stability of Ecosystem faunctions, Drivers of community change
    "Regime shifts + Diversity" - general intro on current findings related to "climate driven ecosystem shifts"
        - %"We find that the rate of turnover in the phytoplankton community becomes faster during this century, that is, the community structure becomes increasingly unstable in response to climate change. Combined with alterations to phytoplankton diversity, our results imply a loss of ecological resilience, with likely knock-on effects on the productivity and functioning of the marine environment." \cite{henson_future_2021}
        - 

        "Here's where we come in..." Talk about how most of the analysis stopped after 2012, time series died out, but there is more data there and particularly looking at community data, we can see a return.
    - what has been done (Phytoplankton - Statistics)?
    - where are the gaps?
    - less studied aspects: Diversity + Phytoplankton Microscopy

    So we would hypothesize, that if wind is driving the ecosystem, then wind would be best predictor of shits.
    Or nutrients, but if it is a climate indices, then the indices capture some variability that is not related to wind or nutrients....
        
    
    "In this paper, we..." 

        %Highlight that I am only looking at Phytoplankton Microscopy data, and a bit of Niskin, but nothing else. Only bottom-up drivers.
        %What are the questions I ask myself?? 1. 



    % INTRO TEXT DUMP:

    Biodiversity is a complex concept composed of genetic, taxonomic, and ecological aspects that can additionally be quantified in temporal and spatial scales. In its basic sense, a measure of diversity should quantify two components: the number of items and their relative frequencies \cite{borics_freshwater_2021}. 
    Richness, the number of species or genera present at a particular time in a sample, has been established as the most basic descriptor of community diversity, but it is not the best indicator of ecosystem change \cite{hillebrand_biodiversity_2018}.
    - Other metrics, evenness, and dominance