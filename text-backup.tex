%OLDCONCLUSION
We found a strong impact of decadal-scale climate variability on phytoplankton community shifts in the tropical coastal ecosystem of the Cariaco Basin. Specifically, phytoplankton cell counts revealed two distinct community clusters, corresponding to periods of high and low upwelling, which aligned with anomalies in the AMO index. We observed no significant differences in overall diversity between the two clusters, except for reduced taxonomic turnover and increased evenness during the low-upwelling phase.
    These community changes did not constitute a permanent regime shift, as a return to the earlier cluster was observed toward the end of the time series (past 2014). Despite multi-annual trends in warming, reduced productivity, and drastic declines in micro-phytoplankton cell counts, the phytoplankton community returned to previous levels once the long-term climatic and local physical conditions were restored. 
    Notwithstanding a number of limitations, our analysis points to the incipient return of the Cariaco Basin to a pre-2004 regime, hinting at an oscillatory nature of phytoplankton communities under global change, an aspect that emphasizes the relevance of long-term time series and deserves further attention. 
    
    The strongest variable predicting changes in the phytoplankton community was the AMO index, highlighting the complex nature of climate-driven ecosystem variability and the connections between global climate and local conditions through low-frequency natural cycles. We concur with \citeA{nye_ecosystem_2014} that the AMO appears to be a suitable candidate as an indicator of ecosystem state, particularly for the phytoplankton community of the Cariaco Basin. 
    Future studies, ideally based on mathematical modeling, could focus on the relative impact of top-down vs. bottom-up drivers, aspects remaining largely unexplored in the Cariaco Basin.
    
    
    

% TOP DOWN ASPECTS FROM INTRO:


During this phase, the heavily exploited sardine fisheries in coastal Venezuela experienced a collapse, coinciding with an increase in zooplankton biomass \cite{taylor_ecosystem_2012}. Interestingly, changes in the phytoplankton community did not result in a significant change in the export of particulate organic carbon (POC) to depth, as measured by sediment traps deployed during the time series \cite{pinckney_phytoplankton_2015}.

The return to conditions similar to those of the late 1990s coincides with a recovery in sardine catches observed in 2016 and 2017 in an area southeast of Isla de Margarita \cite{gomez_gaspar_variacion_2025}. 



%% COASTAL MARGIN FROM INTRO:

Coastal margins, where phytoplankton typically thrive, constitute only \qty{\sim 10}{\%} of the global ocean's surface but are responsible for an estimated \qty{40}{\%} of carbon sequestration to the seafloor \cite{yool_examination_2001, mullerkarger_importance_2005}. Data from remote sensing suggest a global increase in the frequency and spatial extent of coastal phytoplankton blooms over the last 20 years, but regional effects are highly variable \cite{dai_coastal_2023}. Climate change is predicted to have a lasting impact on phytoplankton communities, including shifts in community composition and diversity \cite{acevedo-trejos_glimpse_2014, boyd_biological_2016, henson_future_2021}.


, where community diversity is often highest \cite{brown_why_2014, righetti_global_2019} and faces greater threats of extinction compared to temperate regions \cite{finnegan_paleontological_2015}


The basin is a continental shelf depression comprising two sub-basins of \qty{\sim 1400}{m} depths, which are effectively separated from the Caribbean Sea by a sill with a mean depth of about \qty{100}{m}. The water below the sill is poorly ventilated. This underwater topography, in combination with high surface productivity and aerobic degradation of sinking organic matter, creates permanent anoxic conditions at depths below \qty{\sim 275}{m} \cite{thunell_organic_2000}. Due to these unique properties, the Cariaco Basin has been used as a natural laboratory since the 1950s, providing a rich sedimentary record for the study of past climate \cite{hughen1996nature} and serving as a testbed for understanding the coupling between pelagic production and carbon export to the deep ocean \cite{montes_vertical_2012}. % Describe general seasonal cycle with functional groups etc.


%%%%%%%%%%%%%%%%%%%%%%%%%%%%%%%%%%%%%%%%% BELOW HERE ARE INTRO DRAFT TEXTS! %%%%%%%%%%%%%%%%%%%%%%%%%%%%%%%%%%%%%%%  
    
    % "Changes in plankton biodiversity in response to regional changes in oceangraphic conditions" driven by climate oscillaitons (from REVIEW 2019).
    %% Changes in Functional Groups

    %% Changes in Diversity



    % Now we come into the picture: Which gap are we adressing?

    % FOCUS ON Community Changes + BOTTOM UP


    
    These changes are believed to be influenced by various factors, including climate variability, nutrient fluxes, and ocean circulation patterns.
    % WHAT ARE THE DRIVERS OF COMMUNITY CHANGE? Bottom up OR top down, we focus on bottom up...

    Describe physical changes!

    - AMO (ref to 2019 Review)
    As the CARIACO time series was implemented in 1995, we now know that the Atlantic Multidecal Oscillation (AMO) was shifting to a positive phase for the next 20 years, which resulted in the largest oceanographic anomalies of the last century (McCarthy et al. 2015). These climatic changes had a profound impact throughout the Atlantic (Alexander et al. 2014, Nye et al. 2014) and in the Cariaco basin \cite{taylor_ecosystem_2012,pinckney_phytoplankton_2015}.
    - ENSO teleconnection (ref to 2019 Review)
    The Caribbean experiences the most marked of the El Ni  ̃no–Southern Oscillation teleconnections in the Atlantic Ocean, with warmer sea surface temperatures and trade winds blowing more to the north-northwest during the El Ni  ̃no–Southern Oscillation (Enfield & Mayer 1997).

     - Understanding the dynamics of the Cariaco Basin is crucial to understanding the broader ecological changes in tropical marine ecosystems. + Carbon fluxes in general!


    % Finally say what WE are going to do! What are the open questions, what is new?

    
    %% try to identify drivers of shift in community with Gradient Forest






    %%%%%%%%%%%%%%%%%%%%%%%%%%%%%%%%%%%%%%%%%%%%%%%%%%%%%%%%%%%%%%%%%%%%%%%%%%%%%%%%%%%%%%%%%%%%%%%%%%%%%%%%


    
    %%% Instead of going paper by paper below, I should go topic by topic (i.e. all points made on "DIVERSITY" should be in a paragraph, and all on changes in FGs or Size Classes in another etc...)
    
    Since the inception of the time series, the Cariaco basin has undergone marked shifts...

    The time series includes multiple measurements of the phytoplankton community, which are the bulk fluorometric chlorophyll, microscopy cell counts, and high-performance liquid chromatography (HPLC) pigment analysis. Of these, all cover the entire time series except for the HPLC pigment data, which was performed by three different labs and with a gap between 2000 and 2006 \cite{muller-karger_scientific_2019}. 

    The most often measured indicator of bulk phytoplankton biomass is fluorometric Chlorophyll \textit{a} concentration, which gives a general measure of biomass standing stock and distribution in the water column. A more advanced methodology, high-performance liquid chromatography (HPLC), allows individual pigment types to be quantified. These pigment distributions can be used to estimate the groups of phytoplankton that contribute to biomass. The most specific methodology to quantify phytoplankton communities is the counting and identification of individual phytoplankton cells, either manually under a microscope or automated in flow-through systems. The major drawbacks are the size limit of resolving smaller cells and, if done manually, the complexity of identifying phytoplankton species. 


    % PAPER TIMELINE:
    - Taylor 2012 ()
    \citeA{taylor_ecosystem_2012} analyzed data up to 2010 and found significant decadal trends of increase in sea surface temperature (SST), coinciding with a decrease in upwelling intensity and phytoplankton bloom intensities, which the authors related to the impact of global warming. The phytoplankton community identified via microscopy showed a shift towards smaller taxa after 2004, which coincided with a collapse in commercial sardine landings and an increase in mesozooplankton biomass. The authors explained the change in physical regime in the Cariaco basin with a north-easterly shift of the Inter-Tropical Convergence Zone (ITCZ) and a resulting weakening trend in Trade Winds.
    
    - Mutshinda 2013 ()
    Using a hierarchical Bayesian modeling approach, changes in dominant species quantified via microscopy cell counts between 1995 and 2011 were found to be best predicted by changes in temperature, pH and irradiance, and less so by salinity or macronutrient concentrations \cite{mutshinda_environmental_2013}. In particular, higher temperatures were associated with a lower abundance of diatoms. 
    
    - Irwin 2015 ()
    With the same dataset of microscopy cell counts and a MaxEnt statistical modeling approach, \citeA{irwin_phytoplankton_2015} calculated realized niches for the dominant species for SST, irradiance and nitrate concentration. As trends in the realized niches were observed over the course of the time series, this seemed to show the adaption of species to climate change effects. 
    
    - Pinckney 2015 ()
    The phytoplankton community was investigated in detail by \citeA{pinckney_phytoplankton_2015}, using the microscopy counts and HPLC data from the time ranges 1996 to 2000 and 2006 to 2010. Between these two time periods, they observed that the drastic reduction in diatoms counted using microscopy did not correspond to a similar reduction in diatom abundance calculated from the pigment composition, which is most likely explained by a shift to diatom species with smaller cell sizes. Additionally, the pigment data indicates a deepening of the chlorophyll maximum, an increase in diversity and a reduction in seasonal variability in the community composition. These changes are most likely a response to the reduced upwelling  and a deepening of the euphotic zone associated with a decrease in the strength of seasonal trade winds. Interestingly, the shifts in the phytoplankton community did not result in a significant change to the export of particulate organic carbon (POC) to depth as measured by sediment traps deployed during the time series. 

    - Lorenzoni 2015 ()
    \citeA{lorenzoni_characterization_2015} focused their analysis on HPLC data retrieved between 2006 and 2012 and related seasonal changes in the pigment composition to spectral absorption from remote sensing data and concentrations of organic carbon. They observed a high seasonal variability, with microphytoplankton (mostly diatoms) dominating upwelling periods and picophytoplankton (mostly cyanobacteria) dominating during the rainy season. These community compositions were also reflected in the POC:TChla ratio, which was higher for diatom dominated assemblages and lower for those dominated by cyanobacteria. 
    % "The difference between the taxonomic data and the pigment data suggests that the microscopy taxonomy may not capture the variability in the pico-phytoplankton. Since the beginning of the time-series in 1995, there has been a change in the abundance of phytoplankton size classes based on diagnostic pigments: micro-phytoplankton has decreased year-round, while picophytoplankton has increased, particularly during the rainy season. This is consistent with the decrease in larger phytoplankton previously reported for the region." Lorenzoni et al. 2015
    

    
    - Finally, add the info from Review about Phytoplankton community, about community and upwelling returning in 2014 to previous levels/state.
    - Understanding the dynamics of the Cariaco Basin is crucial to understanding the broader ecological changes in tropical marine ecosystems. + Carbon fluxes in general!

    % Hone in on Diversity Aspects, and what has been done there?
    - Highlight diversity aspect of Pinckney
    - Highlight Mutshinda etc. How they used Niches/Phyto data
    - What are the questions in Diversity TS data? -> Esteban's papers
   

    
    % In recent years, significant changes have been observed in phytoplankton populations within the Cariaco Basin. These changes are believed to be influenced by various factors, including climate variability, nutrient fluxes, and ocean circulation patterns. Understanding the drivers and impacts of these changes in the phytoplankton community is crucial to understand the broader implications for marine food webs and carbon cycling in this and similar ecosystems.
    
    "The Cariaco Basin has undergone marked shifts..." Talk about ecosystem, what has changed, reference prev literature

        % below is from 2019 REVIEW
        %Hydrographic observations (Figure 2) show strong seasonality and interannual variability in the upper 400 m in the Cariaco Basin. This is controlled by seasonal and interannual changes in the wind and in the intensity of the geostrophic Caribbean Current (Muller-Karger et al. 1989). Primary upwelling was strongest between November and May. In general, during upwelling, sea surface temperatures ranged between approximately 23\degree C and 25\degree C. The secondary upwelling that occurs between June and August is shorter ($\sim$ 5 weeks) and is approximately 1.5\degree C warmer than the primary upwelling. It is driven by Ekman transport and an intensification of the curl of the wind close to the southern coast of the Caribbean (Rueda-Roa \& Muller-Karger 2013, Rueda-Roa et al. 2018). Upwelled waters move offshore in plumes that extended as far as 50 km to more than 250 km from the coast (Muller-Karger et al. 1989, 2010). From July to October, there is strong stratification in the upper 50 m of the water column after the wind weakens and the rain and rivers deliver less saline waters (Figure 2).
        %Sea surface temperatures ranged from approximately 28\degree C to 30\degree C during the period of stratification (Figure 2). The mixed-layer depth varied from less than 10 m to 35 m during the 21-year time series (Figure 3). Satellite sea surface temperature and ocean color observations show that upwelling pulsates throughout the year at the scale of days to weeks, leading to marked variation in primary production (Figure 3). CARIACO rigorously characterized the annual seasonal variation in primary productivity and stratification. The length of the time series was key in capturing interannual variations that spanned decades. Satellite sea surface temperature and field data show that upwelling was stronger between 1996 and 2002. After 2002, trade wind intensity became weaker (Taylor et al. 2012).
        The field observations, particularly those in the upper 50 m, showed what appeared to be a return to stronger upwelling conditions starting in 2014 (Figure 2a).


        %“The Caribbean experiences the most marked of the El Nino–Southern Oscillation telecon- ˜ nections in the Atlantic Ocean, with warmer sea surface temperatures and trade winds blowing more to the north-northwest during the El Nino–Southern Oscillation (Enfield & Mayer 1997). ˜” ([Muller-Karger et al., 2019, p. 4](zotero://select/library/items/7YW7MFTU)) ([pdf](zotero://open-pdf/library/items/TTZQBAXW?page=4&annotation=D3H7YG7Y)) link to El Nino of Cariaco/Caribbean sea!

        %“When the CARIACO time series was implemented in 1995, major changes were happening in the Atlantic. We now know that at that time, the Atlantic Multidecadal Oscillation (Kerr 2000, Enfield et al. 2001) started shifting to a positive phase, which then persisted for more than 20 years and led to oceanographic anomalies in the Atlantic that were larger than any recorded over the previous 100 years (McCarthy et al. 2015). The changes had profound biological impacts throughout the Atlantic (Alexander et al. 2014, Nye et al. 2014), including in the Cariaco Basin (Taylor et al. 2012, Pinckney et al. 2015).” (Muller-Karger et al., 2019, p. 3) Interesting aspect that the AMO changed, as apparently the main climate indices to affect the Atlantic/CARIACO
        

    - Why community structure is important, taxonomic diversity (aspects of diversity) - why look at taxonomic diversity?


    - some general intro to DIVERSITY.. what is it? Why important?
    

    % "Phytoplankton is the major primary producer group in marine ecosystems and at the same time a highly suitable indicator for ecosystem change. Due to its high diversity and short generation time, phytoplankton responds quickly to changes in the environment (Reynolds 1998). Especially changes in temperature and nutrient supply are known to change phytoplankton composition and biodiversity, with higher temperatures and lower nutrients favouring small species (Litchman et al. 2007; Winder and Sommer 2012), whereas Si:N and N:P supply ratios change the relative importance of diatoms and other groups such as dino(flagellates) and cyanobacteria (Sommer et al. 2004; Makareviciute-Fichtner et al. 2020)." from Di Cavalho et al. 2023
    

    %"Whereas the positive biomass response to high nutrient and light availability is well known (Tilman 1982; Leibold 1999; Siegel et al. 2013), the biodiversity response is much more complex. First, biodiversity itself is a multifaceted construct, comprising the number of different taxa (e.g. species richness) and the equality of species contribution to the assemblage as aspects of local alpha diversity (evenness). But the change in richness is only the net effect of immigration and local extinction, while the turnover in species composition with time is an additional aspect of biodiversity. Extending to larger spatial scales, heterogeneity in space (beta diversity) and composition of the regional species pool (gamma diversity) are further biodiversity dimensions. Second, for each of these facets, all metrics are effort- and scaledependent. Thus, in contrast to other aspects of ecosystem assessments, biodiversity metrics do not qualify to set absolute target values for good or bad status and effective ecosystem management. But, the relevance of the metrics arises from analysing their temporal trends over a long time period. Third, no single variable captures even the most important aspects of community composition and change. Therefore, the analysis of phytoplankton biodiversity requires a multimetric approach." from Di Cavalho et al. 2023 (with Hillebrand)
    
    - why are we looking at Functional Groups?
    %"The complexity of biodiversity assessment has led to the development of other approaches addressing compositional change in plankton communities, most prominently the creation of functional groups. Functional groups are meant to comprise species that show similar responses to environmental drivers and share morphological and/or physiological characteristics. Functional group indicators have been shown to be relevant for describing community structure and biodiversity and are more comparable with other studies than species-based indicators (Mouillot et al. 2006)." from Di Cavalho et al. 2023 (with Hillebrand)
    
    %% Overarching theme: Stability of Ecosystem faunctions, Drivers of community change
    "Regime shifts + Diversity" - general intro on current findings related to "climate driven ecosystem shifts"
        - %"We find that the rate of turnover in the phytoplankton community becomes faster during this century, that is, the community structure becomes increasingly unstable in response to climate change. Combined with alterations to phytoplankton diversity, our results imply a loss of ecological resilience, with likely knock-on effects on the productivity and functioning of the marine environment." \cite{henson_future_2021}
        - 

        "Here's where we come in..." Talk about how most of the analysis stopped after 2012, time series died out, but there is more data there and particularly looking at community data, we can see a return.
    - what has been done (Phytoplankton - Statistics)?
    - where are the gaps?
    - less studied aspects: Diversity + Phytoplankton Microscopy

    So we would hypothesize, that if wind is driving the ecosystem, then wind would be best predictor of shits.
    Or nutrients, but if it is a climate indices, then the indices capture some variability that is not related to wind or nutrients....
        
    
    "In this paper, we..." 

        %Highlight that I am only looking at Phytoplankton Microscopy data, and a bit of Niskin, but nothing else. Only bottom-up drivers.
        %What are the questions I ask myself?? 1. 



    % INTRO TEXT DUMP:

    Biodiversity is a complex concept composed of genetic, taxonomic, and ecological aspects that can additionally be quantified in temporal and spatial scales. In its basic sense, a measure of diversity should quantify two components: the number of items and their relative frequencies \cite{borics_freshwater_2021}. 
    Richness, the number of species or genera present at a particular time in a sample, has been established as the most basic descriptor of community diversity, but it is not the best indicator of ecosystem change \cite{hillebrand_biodiversity_2018}.
    - Other metrics, evenness, and dominance



    ------


    

- What could have driven these shifts?
% talk about link to bottom up drivers, but also mention top down aspects... then go deep on Gradient Forest analysis, what that could mean. Relate this to other papers about community changes, climate effects and gradient forest applications.


3. Clustering shows the return of the community to the previous cluster, confirmed by monthly NMDS, but EnvFit is interesting, as no variable clearly follows the gradient separating the two clusters (NMDS2).

- Climate indices: AMO stark difference between clusters, cluster 1 is negative anomaly, cluster 2 is positive anomaly, corresponding to effects of AMO -> ITCZ etc.
- MEIv2 less clear, but shows a preponderance of positive anomalies in cluster 1 ... why???
% MEI index relevant (from 2019 REVIEW)
The Caribbean experiences the most marked of the El Niño–Southern Oscillation teleconnections in the Atlantic Ocean, with warmer sea surface temperatures and trade winds blowing more to the north-northwest during the El Niño–Southern Oscillation \cite{enfield_tropical_1997}.

4. Gradient Forest clearly identifies climate indices as best predictors of community change, then nutrients etc.
    This is really interesting, as it would be expected, that in situ-measurements are best predictor of community. BUT either the community + top down processes are capturing more of that variance, or we simply are missing a bottom-up variable that can explain it.
    but that most likely requires mechanistic modeling to investigate further...

    %- Discuss effect of AMO
    % Which studies have already looked at effects of Climate on Ecosystem?
    - AMO has emerged as an important indicator of the climate system and how it affects the ecosystems of the north Atlantic \cite{nye_ecosystem_2014}. The oscillations of the AMO index are strongly correlated with temperature indices in the Caribbean region \cite{stephenson_changes_2014}.
    - Although regional effecs may vary, a positive phase in the AMO generally linked to periods of warmer temperatures and a shift of the ITCZ from north to south.
    
    - Taylor 2012 looked at NAO and MEI, and mostly looked linear trends in variables (they missed the return of the community and regime). NAO is the north-south diplole in sea level pressure, that is primarily governed by atmospheric dynamics \cite{hurrell_north_2009}. NAO and AMO are l
    - AMO is mentioned in 2019 review
    "The observed large-scale multidecadal fluctuations in the Atlantic sea surface temperature (SST), has been referred to as the Atlantic Multidecadal Oscillation (AMO) \cite{kerr2000north} to emphasize the “multidecadal” character of this oceanic phenomenon and to distinguish it from the inter-annual variability associated with the atmospheric North Atlantic Oscillation (NAO) \cite{enfield2001atlantic}." from AMO website

    
    
    %- Discuss effect of MEI
    Since El Nino has strong teleconnection to Caribbean (see ref below from Review) we included this as variable to explain changes in community. From Results it is less pronounced than AMO, but still explains some variability (also ref to NMDS).

    Interestingly, in Gradient Forest Model, Climate variables are the strongest predictors of changes in community (i.e. changes in genus abundances).
    - Discuss what climate variables actually capture, and why they might be better predictors of community changes than local (directly measured) variables.
    

    - AMO (ref to 2019 Review)
    As the CARIACO time series was implemented in 1995, we now know that the Atlantic Multidecal Oscillation (AMO) was shifting to a positive phase for the next 20 years, which resulted in the largest oceanographic anomalies of the last century (McCarthy et al. 2015). These climatic changes had a profound impact throughout the Atlantic (Alexander et al. 2014, Nye et al. 2014) and in the Cariaco basin \cite{taylor_ecosystem_2012, pinckney_phytoplankton_2015}.
    - ENSO teleconnection (ref to 2019 Review)
    The Caribbean experiences the most marked of the El Ni  ̃no–Southern Oscillation teleconnections in the Atlantic Ocean, with warmer sea surface temperatures and trade winds blowing more to the north-northwest during the El Ni  ̃no–Southern Oscillation (Enfield & Mayer 1997).






 
--------------------------------------------------------------------------
----------
----------
----------
----------
----------




- We are not able to fully resolve....
% "Phytoplankton communities exhibit a great deal of variation due to a combination of regulation of growth by bottom-up factors such as resource availability (Grover 1997), variation in mortality due to grazing, viral and parasitoid attack (Alpine & Cloern 1992; Mu ̈ hling et al. 2005), and demographic drift among species (neutral variation). Bottom-up factors can be complex, depending on nutrient supply rates and ratios (Tilman 1982), light availability and photoinhibition (Six et al. 2007; Alderkamp et al. 2010), species-specific competitive ability for resources (Litchman 2007) and other factors influencing physiological responses such as temperature (Eppley 1972). Variation in mortality is much less frequently measured than the factors promoting growth due to the technical challenges in making measurements. None of our environmental variables are good proxies of grazing rates or whether the abundance of a particular population is determined by grazer control or resource supply at a particular time. We have ident" from Mutshinda et al. 2013

%"Taylor et al. 2012 observed that the decline in cell abundances was disproportionate compared with the smaller decrease in total chlorophyll a, which implies that phytoplankton of smaller size were not detected in the surveys.. This was supported by high-performance liquid chromatography analyses, which indicated a reduction in diatom concentrations and an increase in phytoplankton diversity." from 2019 Review 

%"A return to a phytoplankton community structure similar to that observed in the late 1990s was observed in 2014 and 2015 (Figure 5a), as upwelling intensity increased to levels similar to those seen before 2003." from 2019 Review Legacy of CARIACO

- We are analyzing long term time series of data, where probably different analysts have counted the phytoplankton samples...
%It is well known that the personal component has an influence on how well species are identified. Even the same person counting the samples can evolve over time, gaining more experience, which can also influence the results (Löder et al. 2012; Nohe et al. 2018). A recommendation for further improvement of the assessment is a case wise cross-validation of some samples by the monitoring agencies involved. Another important issue to be considered are the differences in sampling frequency between countries and also between stations (see ESM Fig. S1). In highly dynamic and turbid waters such as the Wadden Sea, low frequency and infrequent sampling is not sufficient to properly capture temporal signals (Fettweis et al. 2023). % from Di Cavalho et al. 2023
- We group species and genus counts, to genus level, and use genus counts as variables the model predicts. This assumes that there are not very specific niches etc. within genus, but that they are homogenous --> caveat!
- Discuss limitation of taxonomic diversity, i.e. not direct translation to functional diversity, but here no Data foundation to look at functional diversity.. (inspired by Borics et al. 2021)













These changes are well reflected in the yearly mean anomalies shown in figure \ref{fig:zscore}. After 2004 we see higher SST, lower wind speed, reduced fluorometric chlorophyll \textit{a}, a collapse in functional group cell counts, all matching a shift in the AMO toward positive mean anomalies until 2014.
However, the shift in yearly mean genus richness does not follow the abundances in the cell counts. Instead, we see a marked reduction following 1998 up until 2010, where richness recovers four years before the community seems to return to previous levels in 2014. The same trend is apparent, although to a less pronounced degree, in the other diversity metrics. Generally, mean anomalies in Shannon diversity and Pielou's evenness correlate with genus richness and show a reduced yearly mean between 2000 and 2006, during which time the cell counts registered an increase in nanoflagellates and a pronounced peak in cyanobacteria abundance. 
%(Here check if this is due to "synechococcus" counts, that were abandoned past 2003). 
The period between 2001 and 2006 lacks coverage in HPLC data, so we have to rely on microscopy data to deduce changes in the phytoplankton community. This limits the size range of the observable phytoplankton and thus the measurable diversity. We could deduce that we see a marked reduction in the diversity of microphytoplankton ahead of the collapse in abundance, and similarly an increase in microphytoplankton diversity a few years ahead of the return in abundance to previous levels. 

- major thing is that previous paper only looked at microscopy to 55m (Mutshinda to 75m and Pinckney), but now we look at full 100 m!

- Seasonality and Depth Distribution of biomass and diversity.
The diversity of phytoplankton peaks in the surface ocean, similar to the patterns observed using genetic methods \cite{yeh_contrasting_2022}.



2. Mismatch between Diversity variables, and cell count abundances + community clustering / ecosystem shift

   The time series includes multiple measurements of the phytoplankton community, which are bulk fluorometric chlorophyll, microscopy cell counts, and high-performance liquid chromatography (HPLC) pigment analysis. Of these, all cover the entire time series except for the HPLC pigment data, which was performed by three different labs and with a gap between 2000 and 2006 \cite{muller-karger_scientific_2019}. 

    The most often measured indicator of bulk phytoplankton biomass is fluorometric Chlorophyll \textit{a} concentration, which gives a general measure of biomass standing stock and distribution in the water column. A more advanced methodology, high-performance liquid chromatography (HPLC), allows individual pigment types to be quantified. These pigment distributions can be used to estimate the groups of phytoplankton that contribute to biomass. The most specific methodology to quantify phytoplankton communities is the counting and identification of individual phytoplankton cells, either manually under a microscope or automated in flow-through systems. The main drawbacks are the size limit for resolving smaller cells and, if done manually, the complexity of identifying phytoplankton species. 








    


5. Compare to what happened at other long-term time series in the ocean...







% Pinckney et al. 2015:
%The Cariaco Basin has undergone marked shifts in biogeochemistry and food web structure over the past two decades (Taylor et al., 2012; Montes et al., 2013; Bates et al., 2014; Scranton et al., 2014). Since the initiation of the CARIACO Ocean Time-Series, near-surface (0–55 m) chl a concentrations (measured by fluorescence of pigments extracted in methanol) and net primary production rates (NPP, measured using 14C incubations) have declined, particularly between 2004 and 2010 (Taylor et al., 2012; Scranton et al., 2014). This is hypothesized to be the result of a measured decrease in seasonal trade wind intensity, which has reduced coastal upwelling and led to shallower mixed layer depths (Taylor et al., 2012). Microscopy observations indicate that microphytoplankton assemblages (420 mm) within the euphotic zone have shifted from a community comprised mainly of diatoms, dinoflagellates, and coccolithophorids to one with increasing numbers of smaller taxa including cyanobacteria (Montes et al., 2013). During this period, a collapse in the sardine industry also occurred. The causes are not clear, but the phenomenon was accompanied by an increase in zooplankton biomass and an increase in particulate organic carbon export below the euphotic zone (Taylor et al., 2012). These results suggest that changes in zooplankton abundance and therefore grazing may have contributed to the decline in diatoms and to more efficient carbon export to depth.
% Consistent with the results of Taylor et al. (2012), the microscopy data show large scale reductions in overall microphytoplankton (4 20 mm) cell densities and a large decrease in both diatoms and coccolithophores over the upper 55 m. However, the observed photopigment decline in diatom abundance is not nearly to the same extent and coccolithophores had a significant concentration increase of 60.1% over this same depth interval (Table 3). We argue that the reduction in cell numbers is likely only true of larger cells. Microscopic cell counts are inherently selective and biased toward easily identifiable species and cells that remain intact after preservation (Lund et al., 1958; Rott, 1981; Wilhelm et al., 1991; Schlüter et al., 2000). Furthermore, Mousing et al. (2014) found that small increases in temperature (as observed in the Cariaco Basin) lead to a decrease in the relative contribution of large cells in the community, regardless of ambient nutrient availability. Finally, depth limited microscopy data misses the downward movement of diatoms and coccolithophores deeper in the water column (i.e. below 56 m). In fact, there was also a dramatic shift in diatom speciation, with centric diatoms (Guinardia, Leptocylindrus, Skeletonema) dominating during the first half of the time-series and a mix of centric and pennate diatoms (Pseudonitzschia, Leptocylindrus, Guinardia) dominating in more recent years. In microscopy, smaller sample volumes may lead to underestimations of species richness or diversity (Cermeño et al., 2014). Nonetheless, both photopigment and microscopy measurements suggest that phytoplankton communities became more homogeneously distributed over the years, with the majority of the difference due to declining diatom populations that occur during upwelling events.


% REALLY ONLY LOOK AT BOTTOM UP HERE; No Top Down effects quantified

% AMO


% "Shifts in global and basin-specific primary production and phytoplankton abundances are related to large-scale climatic variability in the physical environment. Despite the lack of a definitive period associated with the AMO, the relationship between the atmospheric variability associated with the NAO and thermohaline circulation characterized by the AMO suggests a coupling of ocean and atmospheric processes on time scales that will facilitate understanding and prediction of inter-annual weather, precipitation, fluvial output and estuarine and marine productivity. Several studies have documented shifts related to the AMO in oceanic (Martinez et al., 2009), coastal and continental shelf (Rodrigues et al., 2009; Schofield et al., 2008) and estuarine (Hubeny et al., 2006) productivity and responses in forage fish populations to this increase in planktonic productivity (Tourre et al., 2007). Correlations between the AMO indices and marine productivity have been documented at immediate (Schofield et al., 2008), historical (Martinez et al., 2009; Tourre et al., 2007), and pre-historical (Hubeny et al., 2006; Rodrigues et al., 2009) scales. There are several mechanisms through which this might occur. Stratification within the water column separates nutrients at lower depths from entering the euphotic zone, thus limiting primary productivity. Long-term climate oscillations may augment or suppress seasonal conditions that promote or erode stratification and regulate mixing. Schofield et al. (2008) have demonstrated links between the AMO and phytoplankton blooms in the Mid-Atlantic Bight, such that increased nutrient availability during the annual fall transition were augmented by increased winter winds associated with a positive phase in the AMO. Martinez et al. (2009) demonstrated that global phytoplankton productivity is linked to multi-decadal oscillations in both the Atlantic and Pacific, suggesting that links between chlorophyll and sea surface temperature are driven by interactions between the pycnocline and mixed surface layers. The influence of large-scale climate fluctuations on surface stratification has particular influence on phytoplankton productivity in the tropics and mid-latitudes, waters which are often nutrient-limited. Alternatively, warmer sea surface temperatures themselves may influence productivity. Shifts in fluvial inputs associated with altered precipitation patterns will also alter river-induced fertilization of marine productivity." from Nye et al. 2014

%"The AMO is also associated with changes in wind and current regimes (Delworth et al., 2007; Häkkinen et al., 2011). Thus, passive dispersal of marine organisms, many of which are planktonic or have planktonic larvae, will be affected resulting in changes in spatial distribution, survival, and population abundance." from Nye et al 2014




%%% DISCUSSION DUMP:

- What we want to mention, but have not included in our analysis due to lack of data at appropriate coverage or resolution, is the indicators of top-down effects on the Cariaco basin ecosystem. Yearly landings of the Spanish sardine (\textit{Sardinella aurita}) fisheries off the NE coast of Venezuela collapsed after 2004 and measurements of zooplankton abundances since 2001 showed a steady increase in numbers up until 2010. These can be viewed as consequences of the changes apparent in the phytoplankton community, but can also be important drivers of that change, for example if the collapse in fisheries was mainly driven by overfishing. (MENTION THAT ZOO DATA IS TOO FEW DATAPOINTS TO INCLUDE IN GRADIENT FOREST MODEL RUN)

FOR THE CLUSTERING (moved from Results): This confirms previously observed patterns in the Cariaco ecosystem, where a strong shift in the community happened in 2004, but the community returned to the previous regime past 2014 \cite{taylor_ecosystem_2012, muller-karger_scientific_2019}.









%%%%%%%%%%%%%%%%%%%%%%%%%%%%%%%%%%%%%%%%%%%%%%%
% KEY POINTS
%%%%%%%%%%%%%%%%%%%%%%%%%%%%%%%%%%%%%%%%%%%%%%%
%  List up to three key points (at least one is required)
%  Key Points summarize the main points and conclusions of the article
%  Each must be 140 characters or fewer with no special characters or punctuation and must be complete sentences

% Example:
% \begin{keypoints}
% \item	List up to three key points (at least one is required)
% \item	Key Points summarize the main points and conclusions of the article
% \item	Each must be 140 characters or fewer with no special characters or punctuation and must be complete sentences
% \end{keypoints}

\begin{keypoints}
\item The CARIACO time series showed a highly variable phytoplankton community over the course of sampling between 1995 and 2017. Clustering of community data from microscopy cell counts reveals two separate clusters, defined by high- and low-upwelling conditions that match the anomalies in the AMO index. 
\index Instead of a unidirectional trend, we see a shift around 2004 to low-upwelling conditions, and a return to the previous community and physical environment in 2014. 
\item Despite the differing physical regimes and the large changes between the functional groups, there are no significant differences in genus richness and Shannon diversity between the two clusters, but community evenness was higher and community turnover was lower in the second cluster (low-upwelling conditions).
\item A gradient forest analysis of community data reveals that the strongest predictor of shifts within the community is the AMO index with a time lag of 2 months, pointing to a strong impact of large-scale climatic oscillations in the tropical coastal ecosystem. The next strongest predictor is sea surface temperature (SST), with the MEI v2 index (lag of 4 months) and nitrate concentration following behind. 
\end{keypoints}





    % Phytoplankton is under threat
    On a global scale, human activities are driving a decline in biodiversity \cite{tittensor_mid-term_2014}. Climate change is predicted to have a lasting impact on phytoplankton communities, shifting the ranges and contributions of functional groups \cite{henson_future_2021}. The drivers and effects of changes in biodiversity need to be further investigated, particularly in the tropics \cite{clarke_does_2017}, where community diversity is often highest \cite{brown_why_2014, righetti_global_2019} and is under greater threat of extinction compared to temperate regions \cite{finnegan_paleontological_2015}.





    %%%%%%%INTRO BACKUP REWRITE MAY "="%%%%%%%%

      % Phytoplankton is important
    Phytoplankton provide the foundation for marine food webs and play a crucial role in biogeochemical processes, including oxygen production and carbon export to the deep ocean \cite{falkowski_biogeochemical_1998}.
    Phytoplankton consists of a diverse set of organisms that exhibit highly variable ecophysiological properties \cite{appeltans_magnitude_2012}. The short generation time and high diversity allow phytoplankton communities to adapt quickly to environmental changes, making them highly suitable indicators of the ecosystem state \cite{alvarez-cobelas_what_1998, barton_anthropogenic_2016, di_cavalho_temporal_2023}.
    The structure of the community can be considered an integrator of environmental factors, as it is strongly influenced by temperature, nutrient availability, and light availability \cite{mouw_phytoplankton_2016}. In addition to these bottom-up effects, influences from higher trophic levels, such as zooplankton, also structure the community and affect biodiversity through grazing pressure \cite{banas_adding_2011}. Disentangling the effects driving changes in community composition and biodiversity of phytoplankton is a challenging but crucial task for better understanding our oceans.

    
    Climate change is predicted to have a lasting impact on phytoplankton communities, shifting the ranges and contributions of functional groups \cite{henson_future_2021}. The drivers and effects of these changes require further investigation, particularly in the tropics \cite{clarke_does_2017}, where community diversity is often highest \cite{brown_why_2014, righetti_global_2019} and faces greater threats of extinction compared to temperate regions \cite{finnegan_paleontological_2015}.
    
    Coastal margins constitute \qty{\sim 10}{\%} of the total ocean surface area, but are responsible for an estimated \qty{40}{\%} of carbon sequestration to the seafloor \cite{yool_examination_2001, mullerkarger_importance_2005}.
    
    - Interestingly, the trend for coastal phytoplankton biomass has been far from diminishing.
    Using remote sensing, a global increase in coastal phytoplankton bloom frequency and spatial extent has been observed \cite{dai_coastal_2023}.

    % We need time series in order to understand effects and changes (threats) 
    Time series of biological observations offer valuable insights into how marine ecosystems respond to temporary shifts, cyclic processes, and long-term changes in the local physical environment and climate, especially when data collection spans several decades \cite{carstensen_need_2014, henson_observing_2016}. 

    
    A tropical coastal ecosystem that has been studied intensively is the CARIACO time series, which has involved consistent sampling of physical and biogeochemical parameters in the Cariaco Basin from 1995 to 2017 \cite{muller-karger_scientific_2019}. 
    
    % Describe CARIACO location
    The Cariaco Basin, located off the coast of Venezuela in the southern Caribbean Sea, is a highly productive ecosystem within the largest truly marine anoxic basin in the world \cite{edgcomb_accessing_2011}. The basin is a continental shelf depression consisting of two sub-basins of \qty{\sim 1400}{m} depths, which are effectively separated from the Caribbean Sea by a sill with a mean depth of about \qty{100}{m}. The water below the sill is poorly ventilated, which, combined with the high surface productivity and microbial respiration of sinking organic matter, results in permanent anoxic conditions at depths below \qty{\sim 275}{m} \cite{thunell_organic_2000}. Due to these unique properties, the Cariaco Basin has been used as a natural laboratory since the 1950s, providing a rich sedimentary record for the study of past climate \cite{hughen1996nature} and serving as a testbed for studying the coupling between pelagic production and carbon export to the deep ocean \cite{montes_vertical_2012}.

    % Describe general seasonal cycle with functional groups etc.
    A markedly seasonal upwelling cycle is driven by easterly trade winds, with the highest levels of primary production observed from December through May when the Inter-Tropical Convergence Zone (ITCZ) moves southward. A smaller wind-driven secondary upwelling often occurs in July and August \cite{mullerkarger_annual_2001, astor_seasonal_2003}. During the upwelling period, the supply of deeper nutrient-rich waters supports high biological productivity, and the phytoplankton community is dominated by diatoms \cite{romero_seasonal_2009}, with dinoflagellates, coccolithophorids, and nanoplankton also contributing. In the low-wind rainy season, the Cariaco Basin exhibits oligotrophic conditions, where the fraction of large phytoplankton cells is diminished, but nano- and picophytoplankton are present in considerable numbers throughout the year \cite{lorenzoni_characterization_2015}.    

    % Describe CARIACO changes in ecosystem
    Since the inception of the time series in 1995, the ecosystem of the Cariaco basin has undergone marked changes. \citeA{taylor_ecosystem_2012} analyzed time series data up to 2010 and found significant decadal trends of increase in sea surface temperature (SST), coinciding with a decrease in upwelling and phytoplankton bloom intensities, which the authors related to the impact of global warming. The phytoplankton community identified by microscopy showed a drastic reduction in Diatom abundance and a shift toward smaller taxa after 2004, which coincided with a collapse in commercial sardine landings and an increase in mesozooplankton biomass.
    The trends in microscopy community data were partially confirmed and expanded by \citeA{pinckney_phytoplankton_2015}. They observed that the drastic reduction in diatoms counted using microscopy corresponded, but to a lesser degree, to diatom abundance calculated from the pigment composition measured by high-performance liquid chromatography (HPLC). This is most likely explained by a shift to diatom species with smaller cell sizes, which are more difficult to detect and identify using light microscopy \cite{lorenzoni_characterization_2015}. In addition, the pigment data indicate a deepening of the chlorophyll maximum, an increase in diversity, and a reduction in seasonal variability in the community composition. These changes are most likely a response to the reduced upwelling and a deepening of the euphotic zone associated with a decrease in the strength of seasonal trade winds. Interestingly, changes in the phytoplankton community did not result in a significant change in the export of particulate organic carbon (POC) to depth, as measured by sediment traps deployed during the time series \cite{pinckney_phytoplankton_2015}.


    %%%%%%%%%%%%%%%%%%%%%%%%%%%%%%%%%%%%%%%%%%%%%%%%%%%%%%%%%%%%%%%%%%%%%%%%%%%%%%%%%%%%%%%%%%%%%%%
    %% ADD Duplicate Points here:
    During the initial phase of the time series (1996–2004), the Cariaco Basin underwent a consistent seasonal upwelling cycle, driven by strong trade winds during winter and spring (December – April) \cite{mullerkarger_annual_2001, astor_seasonal_2003}. In this period, approximately \qty{\sim 70}{\%} of the annual depth-integrated primary production was observed. During upwelling, the phytoplankton community was dominated by microphytoplankton, primarily consisting of diatoms, dinoflagellates, and coccolithophorids \cite{pinckney_phytoplankton_2015}.

    Following 2004, there was a transition in the physical and biogeochemical regime characterized by a significant reduction in trade winds during the upwelling period, which led to elevated sea surface temperatures and decreased mixing between the surface and deep ocean layers \cite{taylor_ecosystem_2012}.

    Microscopy cell counts indicate the collapse of large phytoplankton, particularly diatoms, which is confirmed to a lesser extent by pigment data analysis \cite{pinckney_phytoplankton_2015}. In general, a shift was observed in the phytoplankton community toward smaller cell sizes, along with increased abundances of coccolithophores, cryptophytes, and other phytoflagellates, which are typical of stratified water columns \cite{pinckney_phytoplankton_2015}. During this phase, the heavily exploited sardine fisheries in coastal Venezuela experienced a collapse, coinciding with an increase in zooplankton biomass measured in the Cariaco Basin \cite{taylor_ecosystem_2012}.

    Previous analyses of phytoplankton community data did not include time series data beyond 2011, and \citeA{taylor_ecosystem_2012} rightly concluded that the observed changes cannot "be clearly identified as unidirectional trends driven by anthropogenic climate change or whether they reflect low-frequency natural cycles, such as those driving the AMO."  
    %%%%%%%%%%%%%%%%%%%%%%%%%%%%%%%%%%%%%%%%%%%%%%%%%%%%%%%%%%%%%%%%%%%%%%%%%%%%%%%%%%%%%%%%%%%%%%%
    
    
    %%% Maybe add a point about the Chlorophyll being low and high, HPLC vs Fluorometric?
    The shift in the community, due to the warming trend and the reduction in upwelling intensities, does not appear to be a permanent regime shift. Data from 2012 until the end of the CARIACO time series in 2017, which were not included in previous studies of the phytoplankton community, indicate a return to stronger upwelling intensities and productivity levels similar to those of the late 1990s \cite{muller-karger_scientific_2019}.

    %% Gaps that We Adresse:
    % - More in-depth exploration of microscopy data & diversity time series
    % - Unidirectional trends driven by anthropogenic climate change or low-freq natural cycles, such as AMO?
        %Previous analysis of the phytoplankton community data did not include time series data past 2011 and \citeA{taylor_ecosystem_2012} rightly concluded that the observed changes cannot "be clearly identified as unidirectional trends driven by anthropogenic climate change or whether they reflect low-frequency natural cycles, such as those driving the AMO".
    % - Effect if ENSO on Phyotplankton community test, since previously not observed!
             %This could also be part of the rational of the study. Formulated as question, e.g. "it is still unclear the effects of regional climate phenomena such as ENSO on this shift" - Esteban
        % "\citeA{taylor_ecosystem_2012} found no strong correlation between the MEI v.2 and measurements in the Cariaco basin, despite strong teleconnection effects."

    The El Niño Southern Oscillation (ENSO) has a strong teleconnection to the Caribbean Sea, as during ENSO (indicated by a positive MEI v.2 index) sea surface temperatures rise and trade winds blow more to the north-northwest \cite{enfield_tropical_1997}. \citeA{taylor_ecosystem_2012} found no strong correlation between the MEI v.2 and measurements in the Cariaco Basin, and only \qtyrange{24}{36}{\%} explanation of variance with a time lag of 12 months using linear regression.

    % NOw we come in!
    In this study, we document changes in biodiversity and community structure in the Cariaco Basin between 1995 and 2017 and quantify the effects of bottom-up drivers and climate indices on changes in the phytoplankton community. We focus on phytoplankton cell count data obtained through microscopy, as these data have not been fully analyzed in previous studies. We argue that, despite the shortcomings of microscopy methods, they provide a consistent measure of changes in large phytoplankton. Our analysis emphasizes bottom-up drivers of community changes, as these data are available throughout the time series and at a monthly resolution. This study complements the work of \citeA{taylor_ecosystem_2012} and \citeA{pinckney_phytoplankton_2015} with a more detailed analysis of biodiversity aspects using all data available from the CARIACO time series up to the final cruise in 2017. Understanding the regional dynamics of the Cariaco Basin is crucial to understanding the broader ecological changes in tropical marine ecosystems driven by changes in climate. 





    %%%%%%%DISCUSSION BACKUP REWRITE MAY "="%%%%%%%%


\section{Discussion}

This study demonstrates the highly dynamic nature of a tropical coastal marine ecosystem driven by local environmental variability, which is in turn influenced by large-scale climatic changes. In the Cariaco Basin, decadal-scale climate oscillations have impacted the wind-driven upwelling regime, resulting in confounding effects on the composition and biodiversity of the phytoplankton community.
% - a few sentences on basic seasonal dynamics removed here
These changes are well reflected in the yearly mean anomalies (see Figure \ref{fig:zscore}) and the density distribution of monthly measurements (see Figure \ref{fig:clustering}). After 2004, we observed increased sea surface temperatures (SST), decreased wind speeds, reduced fluorometric chlorophyll \textit{a}, and a collapse in functional group cell counts, all corresponding to a shift in the Atlantic Multidecadal Oscillation (AMO) toward positive mean anomalies until 2014. 
% - a few sentences about shift to smaller taxa and collapse in Sardine Fisheries

% - "We found that including the time series data until the end of 2017 in our analysis clearly shows that the observed shifts were reversible, confirming the observations by \citeA{muller-karger_scientific_2019}."
In 2014, physical conditions in the Cariaco Basin returned to a regime of higher wind speeds, increased upwelling, and elevated nutrient concentrations (see Figure \ref{fig:zscore}). Concurrently, there was an increase in chlorophyll \textit{a} biomass (see Figure \ref{fig:divts} $a$) and a return to a community composition more similar to the period between 1996 and 2004 (see Figure \ref{fig:clustering} $a$ and $c$). This shift to a more productive phytoplankton community after 2014 also coincided with a recovery in sardine captures observed in 2016 and 2017 in a location southeast of Isla de Margarita \cite{gomez_gaspar_variacion_2025}.

When comparing the phytoplankton community between the two clusters, we observe a marked reduction in cell counts, which corresponds to a decreased concentration of chlorophyll \textit{a} biomass measured by fluorometry, particularly in the top \qty{25}{meters} of the water column (see Figure \ref{fig:divts}). When including HPLC pigment data, these patterns are less clear, as the total chlorophyll measurement here actually shows an increase between the periods 1996-2000 and 2006-2010 \cite{pinckney_phytoplankton_2015}, particularly at depths below \qty{55}{meters}. It should be noted that the HPLC data were analyzed by multiple laboratories throughout the time series, with a gap in data coverage between 2000 and 2006. \citeA{pinckney_phytoplankton_2015} could not find a consistent method for quantifying total chlorophyll \textit{a} across different laboratories. Here, we focus on the more consistently sampled fluorometric measurements. Independent spectrophotometric measurements of chlorophyll \textit{a} concentration at \qty{20}{meters} depth off the coast of Margarita Island follow the temporal trend of a reduction in chlorophyll concentration between 2004 and 2014 \cite{gomez_gaspar_variacion_2025}. It should be noted that changes in the composition of the phytoplankton community can affect chlorophyll \textit{a} concentrations, as most measurement methods do not capture only chlorophyll \textit{a} but also other related forms of chlorophyll, which can vary depending on the composition of the community \cite{welschmeyer_fluorometric_1994}. 

Despite some difficulties in reconciling HPLC pigment data and microscopy cell counts, both methods indicate a reduction in diatom abundance after 2004. Cell counts of other functional groups decreased, while HPLC data indicate an increase and redistribution of biomass to depths below \qty{55}{meters} \cite{pinckney_phytoplankton_2015}. The decline in cell counts was not proportional to the more modest decrease in total chlorophyll \textit{a}, suggesting that the phytoplankton community transitioned to smaller individuals that were not captured under the microscope \cite{muller-karger_scientific_2019}. This shift to smaller cell sizes and a reduction in diatoms is consistent with predicted effects of warming and reduced mixing on phytoplankton communities \cite{bopp_response_2005}. Increases in temperature can lead to a decrease in the relative contribution of large cells in the community, irrespective of nutrient availability, but are particularly prevalent when (as observed in the Cariaco Basin) temperatures increase and the nutrient supply decreases \cite{mousing_global_2014}. 

% check sizes of cells, then work this into a paragraph with another sentence!!
If we compare the two parts of the first cluster, the community reorganized, particularly as diatoms shifted from a community dominated by centric diatoms (Guinardia, Leptocylindrus, Skeletonema) to one dominated by pennate diatoms (Pseudonitzschia) \cite{pinckney_phytoplankton_2015}.

However, the shift in yearly mean genus richness does not correspond with the abundances in the cell counts. Instead, we observe a marked reduction following 1998 until 2010, after which richness recovers four years prior to the community seemingly returning to previous levels in 2014. The same trend is evident, albeit to a lesser degree, in the other diversity metrics. Generally, mean annual anomalies in Shannon diversity and Pielou's evenness correlate with genus richness and exhibit a reduced yearly mean between 2000 and 2006, during which cell counts registered an increase in nanoflagellates and a pronounced peak in cyanobacteria abundance. %

The period between 2001 and 2006 lacks coverage in HPLC data, necessitating reliance on microscopy counts to deduce changes in the phytoplankton community. This limitation restricts the observable phytoplankton size range and the measurable diversity through the sampling methodology \cite{cermeno_sampling_2014}. We observe a marked reduction in the genus richness of microphytoplankton ahead of the collapse in abundance, and similarly, an increase a few years prior to the return in abundance to previous levels. 
This pattern results in no significant differences in diversity observed from microscopy cell counts between the two clusters, either by Shannon index or by genus richness (see Figure \ref{fig:zscore} and \ref{fig:divts}).
We observe a significant difference in the Pilou index between the two clusters. Both HPLC pigment and microscopy measurements agree that the phytoplankton community became more evenly distributed during low-upwelling conditions of cluster 2 \cite{pinckney_phytoplankton_2015}. This increase in evenness, as quantified by Pilou's index, is most likely explained by the lack of strong upwelling events and the thereby reduced disturbance of the community. The resulting reduced compositional turnover is also observed in a decrease in interannual species turnover during cluster 2 (see Figure \ref{fig:clustering} $b$). As the productivity and biomass of phytoplankton communities are mainly driven by nutrients supplied through disturbance events, this confirms the previously observed inverse relationship between evenness and community performance \cite{lehtinen_phytoplankton_2017, otero_phytoplankton_2020}.

% NOW let's jump to Gradient Forest and discussing CLIMATE impacts! for a final paragraph, and then we are done here!!
To identify the variables that can best predict changes in the phytoplankton community, we conducted a gradient forest analysis using microscopy cell counts at the genus level as predicted variables and climate indices, along with in situ measurements of bottom-up drivers, as predictors. Limiting the analysis to bottom-up drivers ensured the greatest possible data coverage throughout the CARIACO time series. Interestingly, our analysis revealed that the strongest predictors of changes within the community are not in situ measurements, but rather the AMO and the MEI v.2 climate indices. These are followed in importance by nitrate concentration and SST (see Figure \ref{fig:GF}).

These results match the observed NMDS clustering of the monthly cell counts in relation to the predictor variables, since the scaling axis that separates the two yearly aggregate community clusters is NMDS 1, which covaries most strongly with changes in the AMO, MEI v.2 (see Figure \ref{fig:clustering} $c$). Variables directly related to upwelling conditions, such as temperature, \qty{21}{\celsius} isotherm, wind speed, and nutrient concentrations vary more strongly along NMDS2. This indicates that the climate indices capture some variability, which is not represented in the more direct measurements of upwelling intensity. Salinity and silicate concentration, which are also strongly related to water mass exchanges through upwelling, but also depend on precipitation and fresh water input through surface runoff, show greater variance along NMDS1. Both salinity and silicate concentration show some explanatory power in the gradient forest analysis, although they are weaker predictors than temperature, nitrate concentration, and the climate indices. 

Using microscopy cell counts as a basis for this analysis meant that the results should be interpreted in terms of changes in the microphytoplankton community, due to the size limitation of identifying smaller species via light microscopy. Grouping cell counts by genera does create a loss of resolution, but is based on the assumption that species within a genus generally show similar ecophysiological responses. The output of the gradient forest model is further grouped by broad functional types to facilitate the identification of shifts and trends. The phytoplankton genera that can be best predicted are mostly diatoms, but also some haptophytes, dinoflagellates, and cyanobacteria (see Figure \ref{fig:GF} $b$). Within these groups, we see differing threshold responses in cumulative importance along the predictor ranges.

%AMO has emerged as an important indicator of the climate system and how it affects the ecosystems of the north Atlantic \cite{nye_ecosystem_2014}. The oscillations of the AMO index are strongly correlated with temperature indices in the Caribbean region \cite{stephenson_changes_2014}. % COPIED TO INTRO
Although regional effecs may vary, a positive phase in the AMO is generally associated with periods of warmer temperatures and a shift of the ITCZ from north to south. This shift of the ITCZ southward increases wind-driven upwelling in the Cariaco basin \cite{taylor_ecosystem_2012}. The threshold responses between functional groups, which relate to a shift in the cell count abundances, show contrasting patterns. Diatoms and cyanobacteria show strong response thresholds for negative AMO anomalies. This is an interesting threshold response shown for the AMO, which, if sufficiently negative, seems to correlate to a reorganization of the physical regime in the Cariaco basin. The abundance of haptophytes shows a strong threshold at positive AMO anomalies, which could be related to the pattern observed from HPLC data of haptophytes mostly exhibiting blooms above the mixed layer and at a lower frequency from other functional groups \cite{pinckney_phytoplankton_2015}.

% Move the general point on ENSO to intro! Only Discussion-relevant part here..
%copied to Intro: The El Niño Southern Osciallation (ENSO) has a strong teleconnection to the Caribbean Sea, since during ENSO (indicated by a positive MEI v.2 index) sea surface temperatures rise and trade winds blow more to the north-northwest \cite{enfield_tropical_1997}. \citeA{taylor_ecosystem_2012} found no strong correlation between the MEI v.2 and measuerments in the Cariaco basin, and only \qtyrange{24}{36}{\%} explanantion of variance with a time lag of 12 months using linear regression. In contrast, we found a relatively strong predictive power for changes in the phytoplankton community with a 4-month time lag. We also tested a 12 month lag and found no significant improvement.
The threshold responses show a shift in the abundances of dinoflagellates and haptophytes for a negative MEI v.2 index, which indicates La Niña conditions. For cyanobacteria and haptophytes, we found a pronounced threshold response for positive anomalies showing a sensitivity to ENSO events. Diatoms do not show a clear threshold response for the MEI v.2 index, but are affected across the entire range. Positive anomalies of the MEI v.2, representing strong ENSO events, are interestingly only found in Cluster 1 (see Figure \ref{fig:clustering} $e$). 

The in-situ measurements that best predict changes in the microphytoplankton community were nitrate concentration, SST and salinity, in that order. Nitrate is a central nutrient for phytoplankton growth and has been found to be the growth-limiting nutrient during stratification periods in the Cariaco Basin \cite{muller-karger_scientific_2019}. For dinoflagellates and diatoms, we see a strong threshold response around \qty{7}{\micro \mole}, which would indicate that strong upwelling conditions drive blooms of these functional groups. Other functional groups, including \textit{Mesodinium} and \textit{Eutreptria} also show a threshold response at even higher nitrate concentrations (around \qty{9}{\micro \mole}), indicating an effect on their abundance from very strong upwelling. Similarly, for temperature, dinoflagellates and diatoms exhibit a threshold around \qtyrange{21}{23}{\celsius}, which indicates higher abundances during low temperatures in the upper \qty{100}{m} driven by strong upwelling. For the other predictors, we do not observe strong threshold responses.

Gradient forest analysis clearly identified the AMO and MEI v.2 climate indices as the best predictors of changes in the microphytoplankton community in the Cariaco basin. This is an interesting result, as one would expect that more direct measures of ecosystem variability could better predict shifts in abundances. One possible explanation is that in situ measurements contain much stronger variability due to measurement error, local and temporal variability, and generally only provide a single snapshot for an entire month. Climate indices are detrended and integrated measurements of long-term changes that are matching in scale to the dynamics of long-term changes within a phytoplankton community. However, we do not see such strong predictive power in the ERA5 climate reanalysis-derived mean wind speed, which also aggregates measurements and model output over an entire month. 
Another explanation could be that the climate indices capture a dimension of variability not found in our in-situ measurements, which is also represented in our NMDS clustering (see Figure \ref{fig:clustering} $c$), where both MEI v.2 and AMO scale along the second axis (NMDS 2). %TODO test more of my variables along these axes, then add to discussion here and maybe put plot in supplemental material.

% DEFINITELY NEED TO DISCUSS PROBLEMS OF CORRELATION WITH SUCH TIME SERIES DATA

% FROM FRAKER ET AL. 2022
%"Note that although this analysis identifies predictive relationships between drivers and the biological indicators, these relationships do not necessarily imply mechanistic relationships. As such, this analysis represents the first stages of an overall analysis (a narrowing of the potential key relationships and when they matter from a much larger dataset), and additional statistical analyses, modeling, and experimental work may be required to develop a full mechanistic understanding of driver-response relationships and ecosystem change."

- alongside a global reduction in frequency compared to the linear trend between 2006 and 2012 \cite{dai_coastal_2023}. % <- for this paper, check references to ENSO MEI v.2!!! This could be interesting!


Our limitation to bottom-up drivers of community change could also limit the explanatory power of our analysis, as top-down processes such as zooplankton grazing strongly affect the structure of the phytoplankton community \cite{banas_adding_2011, acevedo-trejos_mechanisms_2015}. In the Cariaco basin in particular, a trophic cascade from the collapse of sardine fisheries through an increasing abundance of mesozooplaknton has been proposed to have affected the phytoplankton community \cite{muller-karger_scientific_2019}. Due to the lack of temporal resolution for higher trophic levels and the lack of zooplankton data for the first 6 years of the time series, we propose that mechanistic modeling studies could be well suited to investigate this aspect of the CARIACO time series further.

In this study, we have found a strong impact of decadal-scale climatic oscillations on shifts in the phytoplankton community in the tropical coastal ecosystem of the Cariaco basin. Despite multiannual trends in warming, reduced productivity and a collapse in fisheries and phytoplankton cell counts, the phytoplankton community returned to previous levels once the climatic and local physical conditions returned. We see these patterns clearly in the community data retrieved from microscopy, which shows the value of the arduous work of identifying and counting phytoplankton cells. The strongest bottom-up variable predicting changes in the phytoplankton community is the AMO index, highlighting the complex nature of climate-driven ecosystem variability and the connections between global climate and local ecosystems through low-frequency natural cycles. We hope that the work hints at the resilient nature of phytoplankton communities under global change, which should be further investigated by studying long-term ocean time series.

% Highlight that the climate indices explain changes in the community and abundance, but not necessarily changed in diversity (as there is no sig diff beteween clusters). Probably need more data, other approaches to untangle effects on diversity.




